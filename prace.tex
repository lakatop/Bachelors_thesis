%%% Hlavní soubor. Zde se definují základní parametry a odkazuje se na ostatní části. %%%

%% Verze pro jednostranný tisk:
% Okraje: levý 40mm, pravý 25mm, horní a dolní 25mm
% (ale pozor, LaTeX si sám přidává 1in)
\documentclass[12pt,a4paper]{report}
\setlength\textwidth{145mm}
\setlength\textheight{247mm}
\setlength\oddsidemargin{15mm}
\setlength\evensidemargin{15mm}
\setlength\topmargin{0mm}
\setlength\headsep{0mm}
\setlength\headheight{0mm}
% \openright zařídí, aby následující text začínal na pravé straně knihy
\let\openright=\clearpage

%% Pokud tiskneme oboustranně:
% \documentclass[12pt,a4paper,twoside,openright]{report}
% \setlength\textwidth{145mm}
% \setlength\textheight{247mm}
% \setlength\oddsidemargin{14.2mm}
% \setlength\evensidemargin{0mm}
% \setlength\topmargin{0mm}
% \setlength\headsep{0mm}
% \setlength\headheight{0mm}
% \let\openright=\cleardoublepage

%% Vytváříme PDF/A-2u
\usepackage[a-2u]{pdfx}

%% Přepneme na českou sazbu a fonty Latin Modern
\usepackage[czech]{babel}
\usepackage{lmodern}
\usepackage[T1]{fontenc}
\usepackage{textcomp}

%% Použité kódování znaků: obvykle latin2, cp1250 nebo utf8:
\usepackage[utf8]{inputenc}

%%% Další užitečné balíčky (jsou součástí běžných distribucí LaTeXu)
\usepackage{amsmath}        % rozšíření pro sazbu matematiky
\usepackage{amsfonts}       % matematické fonty
\usepackage{amsthm}         % sazba vět, definic apod.
\usepackage{bbding}         % balíček s nejrůznějšími symboly
			    % (čtverečky, hvězdičky, tužtičky, nůžtičky, ...)
\usepackage{bm}             % tučné symboly (příkaz \bm)
\usepackage{graphicx}       % vkládání obrázků
\usepackage{fancyvrb}       % vylepšené prostředí pro strojové písmo
\usepackage{indentfirst}    % zavede odsazení 1. odstavce kapitoly
\usepackage{natbib}         % zajištuje možnost odkazovat na literaturu
			    % stylem AUTOR (ROK), resp. AUTOR [ČÍSLO]
\usepackage[nottoc]{tocbibind} % zajistí přidání seznamu literatury,
                            % obrázků a tabulek do obsahu
\usepackage{icomma}         % inteligetní čárka v matematickém módu
\usepackage{dcolumn}        % lepší zarovnání sloupců v tabulkách
\usepackage{booktabs}       % lepší vodorovné linky v tabulkách
\usepackage{paralist}       % lepší enumerate a itemize
\usepackage{xcolor}         % barevná sazba

%%% Údaje o práci

% Název práce v jazyce práce (přesně podle zadání)
\def\NazevPrace{Analyzátor USB paketů}

% Název práce v angličtině
\def\NazevPraceEN{USB Packet Analyzer}

% Jméno autora
\def\AutorPrace{Peter Lakatoš}

% Rok odevzdání
\def\RokOdevzdani{2021}

% Název katedry nebo ústavu, kde byla práce oficiálně zadána
% (dle Organizační struktury MFF UK, případně plný název pracoviště mimo MFF)
\def\Katedra{Katedra distribuovaných a~spolehlivých systémů}
\def\KatedraEN{Department of~Distributed and~Dependable Systems}

% Jedná se o katedru (department) nebo o ústav (institute)?
\def\TypPracoviste{Katedra}
\def\TypPracovisteEN{Department}

% Vedoucí práce: Jméno a příjmení s~tituly
\def\Vedouci{Mgr.~Pavel Ježek,~Ph.D.}

% Pracoviště vedoucího (opět dle Organizační struktury MFF)
\def\KatedraVedouciho{Katedra distribuovaných a~spolehlivých systémů}
\def\KatedraVedoucihoEN{Department of~Distributed and~Dependable Systems}

% Studijní program a obor
\def\StudijniProgram{Informatika}
\def\StudijniObor{Programování a~softwarové systémy}

% Nepovinné poděkování (vedoucímu práce, konzultantovi, tomu, kdo
% zapůjčil software, literaturu apod.)
\def\Podekovani{%
Poděkování.
}

% Abstrakt (doporučený rozsah cca 80-200 slov; nejedná se o zadání práce)
\def\Abstrakt{%
USB zbernica je dnes jedným z najrozšírenejších spôsobov pripojenia periférií k počítaču. 
Cieľom práce bolo vytvoriť software, ktorý analyzuje zachytenú komunikáciu medzi zariadnením pripojeným na danú zbernicu a počítačom.

Aplikácia následne rozumným spôsobom vizuálne zobrazuje zanalyzované dáta. Počiatočná verzia sa špecificky zameriava na HID triedu zariadení a ponúka aj sémantický význam jej úzkej podmnožiny do ktorej patria myš, klávesnica a joystick. Pri vizuálnej reprezentácii dát sa práca sa inšpiruje rôznymi dostupnými softwarmi, pričom rozlične kombinuje resp. dopĺňa ich vlastnosti a implementuje z nich tie, ktoré vníma ako najlepšie riešenie v danej situácii.

Ďalšie vlastnosti aplikácie sú napríklad parsovanie HID Report Descriptoru vďaka ktorému je jednoduchšie pridať sémantickú analýzu rôznym ďalším HID zariadeniam. Celkový návrh aplikácie by mal ponúknuť možnosť budúcej implementácie ďalších USB tried pre prípadné rozšírenie.
}
\def\AbstraktEN{%
Abstract.
}

% 3 až 5 klíčových slov (doporučeno), každé uzavřeno ve složených závorkách
\def\KlicovaSlova{%
{USB} {HID}
}
\def\KlicovaSlovaEN{%
{key} {words}
}

%% Balíček hyperref, kterým jdou vyrábět klikací odkazy v PDF,
%% ale hlavně ho používáme k uložení metadat do PDF (včetně obsahu).
%% Většinu nastavítek přednastaví balíček pdfx.
\hypersetup{unicode}
\hypersetup{breaklinks=true}

%% Definice různých užitečných maker (viz popis uvnitř souboru)
\include{makra}

%% Titulní strana a různé povinné informační strany
\begin{document}
\include{intro/titulka}

%%% Strana s automaticky generovaným obsahem bakalářské práce

\tableofcontents

%%% Jednotlivé kapitoly práce jsou pro přehlednost uloženy v samostatných souborech
\chapter{Úvod}

Možnosti pripojenia rôznych periférií k zariadeniu sú v dnešnej dobe rozsiahle. Aj napriek tomu, že technológia každým dňom napreduje a svet sa uberá viac bezdrôtovým smerom, je USB stále najrozšírenejší sériový spôsob prenášania dát. Už z názvu \uv{Universal Serial Bus} je jasné, že rozpätie zariadení, ktoré možno k tejto zbernici pripojiť je obrovské. Práve preto je USB protokol jeden z najkomplexnejších protokolov ktoré sa využívajú na komunikáciu.

V tejto práci sa pozrieme na užšiu podmnožinu USB protokolu a na komunikáciu s dopredu vybratými perifériami, konkrétne myšou, klávesnicou a joystickom.

\section{USB}
vysvetlenie zakladnych pojmov spojenych USB: historia, usb port/conector, plug and play(https://docs.microsoft.com/en-us/windows-hardware/drivers/kernel/introduction-to-plug-and-play), host-master, low/full/high speed zariadenia

\section{Existujúce aplikácie}

Momentálne existuje niekoľko známych aplikácií ktoré slúžia na analýzu USB paketov. V tejto kapitole si ich zopár ukážeme, pričom mnohé z nich nám poslúžili ako inšpirácia pri ppísaní našej práce a riešení konkrétnych problémov na ktoré sa pozrieme v nasledujúcich kapitolách. 

Je nutné upozorniť, že väčšina dnešných analyzátorov sú platené aplikácie, prípadne majú odomknuté len základné vlastnosti s možnosťou dokúpenia si plnej verzie. Práve preto pri ich prípadnom porovnávaní budeme brať do úvahy len funkcionalitu, ktorá je dostupná zadarmo.

\subsection*{Wireshark}

Pravdepodobne najznámejšia third-party aplikácia na analýzu paketov. Jeho funkcionalita je veľmi rozsiahla, a vzhľadom na to, že sa jedná o open-source projekt, neustále rastie. Zameriava sa hlavne na analýzu sieťových paketov. Napriek tomu podporuje spoluprácu s rôznymi inými sniffermi. Jeden z takýchto snifferov je USBPcap ktorý zachytáva USB komunikáciu, a tak nie je prekvapivé, že Wireshark podporuje analýzu paketov aj nad touto zbernicou. 

Vzhľadom na obľúbenosť a rozsiahlosť programu, nám Wireshark poslúžil ako referenčná aplikácia, z ktorej sme čerpali celkovú inšpiráciu na funkcie, ktoré by mal bežný analyzátor paketov spĺňať. Medzi tie úplne základné patrí napríklad hexdump dát nad ktorými prebieha analýza, ale napríklad aj spôsob vyobrazenia rôznych deskriptorov, ktoré je riešené cez stromovú štruktúru. Medzi viac špecifické funkcie ktoré sme neskôr implementovali aj v našom programe patrí napríklad detailnejšie vyobrazenie jednotlivých bytov a ich význam, ako je možné vidieť nižšie na obrázku~\ref{obr:uvod:byte_detail_foto}. Túto vlastnosť aj napriek jej využitiu mnohé konkurenčné aplikácie postrádajú.

Jeho výhoda je hlavne v tom, že podporuje širokú škálu deskriptorov a plná verzia programu je dostupná úplne zadarmo. Z pohľadu užívateľa je až prekvapivé, že aj napriek rozsiahlosti programu je aplikácia veľmi user-friendly orientovaná a dopĺňa ju veľmi intuitívne užívateľské rozhranie.

\begin{figure}
	\centering
	\includegraphics[width=\textwidth]{img/uvod_byte_detail}
	\caption{Ukážka vyobrazenia jednotlivých bytov.}
	\label{obr:uvod:byte_detail_foto}
\end{figure}

\subsection*{Device Monitoring Studio}

Plná verzia aplikácie je platená. Verzia zadarmo ponúka analýzu sieťových a USB paketov, tak ako aj analýzu komunikácie prebiehajúcej cez sériový port. 

Ako prvé na aplikácii zaujme spôsob zvolenia si zariadenia s ktorým bude sledovaná komunikácia. Je implementovaný štýlom stromovej štruktúry ako je ukázané na obrázku~\ref{obr:uvod:treeview_foto} nižšie. Už základná verzia programu poskytuje veľmi rosiahlu funkcionalitu. Používateľovi je umožnené posielať zariadeniu požiadavky definované v USB špecifikácii(ODKAZ). Aplikácia taktiež zobrazuje sémantickú analýzu niektorých HID zariadení, ale nie v práve najlepšej podobe. 

Na základe výstupu sa dá povedať, že sémantická analýza je naimplementovaná skôr obecne a pri niektorých položkách je namiesto ich významu napísané \uv{Unknown}. Taktiež nie je veľmi jasné odkiaľ sa dané hodnoty berú, keďže k nim chýba ich dátová reprezentácia. Medzi zachytenými paketami sa prvotne nezobrazujú tie, ktoré označujú nakonfigurovanie daného zariadenia (je nutné ho odpojiť a znova napojiť počas monitorovania). Užívateľské rozhranie je veľmi chaotické a chvíľu trvá, kým človek nájde čo i len základné informácie ako napríklad hlavičky ku jednotlivým paketom.  Nepoteší ani fakt, že verzia zadarmo nedovoľuje monitorovanie dlhšie ako 10 minút a maximálny počet monitorovaní za jeden deň je taktiež 10.

\newpage

\begin{figure}[!htb]
	\centering
	\includegraphics[width=\textwidth]{img/uvod_treeview}
	\caption{Ukážka stromovej štruktúry na zvolenie si zariadenia s ktorým bude zachytávaná komunikácia.}
	\label{obr:uvod:treeview_foto}
\end{figure}

\section{Požadované funkcie}
Na základe minulých príkladov už existujúcich aplikácií sme si mohli všimnúť, že všetky obsahujú niekoľko základných funkcií, ktoré by mal obsahovať každý analyzátor. V tejto sekcii zhrnieme funkcie, ktoré sme si zvolili pre našu aplikáciu. Zo základných funkcií sme si vybrali tie, ktoré považujeme za absolútne nevyhnutné pre každý analyzátor. Následne sme sa niekoľko z nich pokúsili akýmsi spôsobom vylepšiť tak, aby maximalizovali ich využiteľnosť a zároveň čo najlepšie zlepšovali prácu s aplikáciou jej užívateľovi. Zároveň ich dopĺňame o pokročilejšie funkcie z ktorých sú niektoré viac špecificky zamerané.

\subsubsection*{Základné}

Výsledná aplikácia by teda mala splňovať nasledujúce zákadné požiadavky:

\begin{enumerate}[label=\textbf{P\arabic*}]
	\item Mala by byť schopná analyzovať USB pakety zachytené v \textbf{pcap} formáte pomocou \textbf{USBPcap} snifferu.
	\item Mala by podporovať sémantickú analýzu pre všetky základné USB descriptory spomenuté v USB 2.0 dokumentácii(ODKAZ kapitola 9.6) (ako napríklad \textit{Device descriptor}, \textit{Interface descriptor}, \textit{Endpoint descriptor},...)
	\item Mala by vedieť rozumne zobraziť dáta ktoré \textbf{USBPcap} zachytí a uloží. Pod pojmom rozumne myslíme spôsob zobrazenia pomocou hexdumpu.
	\item Mala by na prvý pohľad jasne zobraziť základné informácie o každom analyzovanom pakete (ako napr. dĺžka paketu, typ prenosu, ...) a pri bližšom skúmaní jednotlivých paketov detailnejšie zobraziť celú jeho hlavičku.
	\item Detailnejšie informácie o pakete budú zobrazované na základe interakcie užívateľa s aplikáciou.
\end{enumerate}

\subsubsection*{Pokročilejšie}

Zároveň sme aplikáciu doplnili o niekoľko pokročilejších a špecificky zameraných požiadaviek:

\begin{enumerate}[label=\textbf{P\arabic*},resume]
	\item Mala by určitým spôsobom uľahčiť používateľovi orientáciu v hexdumpe.
	\item Mala by byť schopná rozparsovať \textbf{HID Report Descriptor} takým štýlom, aby bolo neskôr možné sématnicky reprezentovať input určitých HID zariadení
	\item Mala by byť schopná vhodným spôsobom vizuálne zobraziť sémantický význam dát posielaných danou podmnožinou HID zariadení do ktorej patrí myš, klávesnica a joystick.
	\item V miestach kde to dáva zmysel, by aplikácia mala byť schopná zobrazovať význam dát až na úrovni jednotlivých bitov.
\end{enumerate}

\section{Ciele práce}
vytvorit aplikaciu
navrh musi byt dostatocne obecny aby sa dal rozsirit o dalsie USB protokoly
dostatocne obecny navrh pre lahke pridavanie novych HID zariadeni
prehladny interface








\chapter{USB a Windows}
\section{USB zbernica}
Plug and Play device tree(sposob akym si windows udrziava strom zariadeni na zbernici)(https://docs.microsoft.com/sk-sk/windows-hardware/drivers/gettingstarted/device-nodes-and-device-stacks)
\section{Device object a device stack}
PDO,FDO, Device object(https://docs.microsoft.com/en-us/windows-hardware/drivers/kernel/introduction-to-device-objects)
https://docs.microsoft.com/en-us/windows-hardware/drivers/kernel/creating-a-device-object
\subsection{Drivery}
windows driver model(WDM) : https://docs.microsoft.com/en-us/windows-hardware/drivers/kernel/types-of-wdm-drivers
bus driver(https://docs.microsoft.com/en-us/windows-hardware/drivers/kernel/bus-drivers), function driver(https://docs.microsoft.com/en-us/windows-hardware/drivers/kernel/function-drivers) a filter driver(https://docs.microsoft.com/en-us/windows-hardware/drivers/kernel/filter-drivers)
\section{Komunikacia s USB zariadenim}
sposob komunikacie operacneho systemu so zariadenim pripojenym na USB zbernicu : IRP(https://docs.microsoft.com/en-us/windows-hardware/drivers/gettingstarted/i-o-request-packets) , URB (https://docs.microsoft.com/en-us/windows-hardware/drivers/usbcon/communicating-with-a-usb-device) a pod.
https://docs.microsoft.com/en-us/windows-hardware/drivers/kernel/handling-irps
\section{USB descriptory}
opis zakladnych USB descriporov, hlavne tych ktore neskor aj vyuzivam v program(Device, Interface, Endpoint, Configuration, String, Setup) : https://docs.microsoft.com/en-us/windows-hardware/drivers/usbcon/usb-descriptors
https://docs.microsoft.com/en-us/windows-hardware/drivers/usbcon/usb-control-transfer
\subsection{Rozlozenie USB zariadenia z hladiska descriptorov}
https://docs.microsoft.com/en-us/windows-hardware/drivers/usbcon/usb-device-layout
\section{HID zariadenia}
hid zariadenie obecne, priklady
https://docs.microsoft.com/en-us/windows-hardware/drivers/hid/
\subsection{Reporty}
Input/Output/Feature reporty.
\subsection{Report Descriptor}
Opis report descriptoru, k comu sluzi, pripadne ako z neho vycitat zaujimave data (neskor vyuzite v programe pri parsovani HID Report Descriptoru na naslednu semanticku analyzu dat ktore posiela zariadenie)






\chapter{Analýza}
\section{Získanie USB packetov}
Na získavanie USB paketov nám bude obecne slúžiť paket sniffer. Väčšina paket analyzátorov má implementované vlastné sniffery a preto sme sa o to pokúsili tiež. Narazili sme ale na niekoľko zásadných problémov, ktoré sa úzko viažu s platformou na ktorú cielime s našou aplikáciou~--~Windows.

Microsoft dokumentácia podrobnejšie opisuje komunikáciu medzi HID zaridením a kernel/user-mode aplikáciou~\cite{hid_opening_collections}. Keďže naša aplikácia beží v user-mode, prejdeme si tento spôsob:
\begin{enumerate}
\item Aplikácia nájde a identifikuje HID zariadenie.
\item Aplikácia pomocou metódy \textit{CreateFile} otvorí spojenie s HID zariadením.
\item Aplikácia pomocou \textit{HID API}~\cite{hid_api} metód \textit{HidD\_Xxx} získa \textit{Preparsed Data} a informácie ohľadom HID zariadenia.
\item \textbf{Aplikácia použije metódu \textit{ReadFile} resp. \textit{WriteFile} na získanie inputu zariadenia resp. poslanie reportu zariadeniu.}
\item Aplikácia pomocou \textit{HID API}~\cite{hid_api} metód \textit{HidP\_Xxx} interpretuje HID reporty.
\end{enumerate}

\subsection{Windows exclusive mód}
Windows má definovaný tzv. \textit{Access Mode}, ktorý určuje restrikciu prístupu \textit{HID Clienta} k HID zariadeniu. 
Ten môže byť buď \textit{Shared} alebo \textit{Exclusive}. \textit{Exclusive Mode} zabraňuje ostatným \textit{HID Clientom} v zachytávaní alebo získavaní inputu HID zariadenia, pokiaľ nie sú hlavným príjemcom daného inputu. Preto z bezpečnostných dôvodov otvára \textit{RIM (Raw Input Manager)} niektoré zariadenia v \textit{Exclusive Mode}.

Ak je zariadenie otvorené v \textit{Exclusive Mode}, aplikácia má stále prístup k niektorým jeho údajom pomocou  \textit{HID API}~\cite{hid_api} metód  \textit{HidD\_\textbf{Get}Xxx}. Tieto metódy nám obecne umožnia získať niektoré descriptory zariadenia, tak ako aj jeho \textit{Preparsed Data}. Nie je nám ale umožnené volať metódu \textit{ReadFile}, takže nemáme akým spôsobom zachytávať komunikáciu HID zariadenia s clientom.

Tabuľka zariadení~\cite{hid_access} (obrázok~\ref{obr:kap3:access_mode}), ktoré \textit{RIM} otvára v \textit{Exclusive Mode} obsahuje aj tie, ktoré sme si v úvode zvolili ako podmnožinu HID zariadení na analýzu -- myš a klávesnica.

\begin{figure}[!htb]
	\centering
	\includegraphics[width=\textwidth]{img/kap3_access_mode}
	\caption{Tabuľka zariadenía ich \textit{Access Mode}. Zariadenia postupne po riadkoch -- myš, joystick a klávesnica}
	\label{obr:kap3:access_mode}
\end{figure}

\newpage

\subsection{Známe knižnice}
opisat zakladne kniznice na sledovanie USB zbernice a preco som ich nemohol pouzit : libUSB, hidAPI

\subsection{Driver}
TU povedat riesenie - pouzitie driveru na komunikaciu so zariadenim. Existujuce windows drivery -- moufiltr, Kbdfiltr - nefunguju pre USB

TU spomenut posledne mozne riesenie - napisanie vlastneho filter driveru. 

\subsection{Third-party aplikácie}
opisat odkial nakoniec ziskavam packety - USBPcap a Wireshark









\section{Spracovávanie pcap súborov}
moznosti ako citat pcap subory : bud pouzit uz existujucu kniznicu : na linuxe Libpcap, windows NPcap(deprecated WinPcap), alebo citat subory manualne : std::istream alebo QFile
\section{Sémantická reprezentácia dát}
ako si z dat vytiahnut udaje ktore su potom pouzite na semanticku analyzu implementovanych HID zariadeni : HID Report parser, InputValues a EndpointDevice struct.
Nasledne sparovanie - ako vybrat spravny report pre konkretny input
\section{Voľba frameworku}
obecne co by som od toho GUI priblizne chcel, potom opisat preco som si vybral prave Qt a v nasledujucich kapitolach opisat rozhodnutia uz v Qt
dovod preco som si zvolil qt namiesto inych c++ GUI frameworkov(napriklad sfml)
\section{Zobrazenie základných informácií}
ako zobrazovat zakladne info o packete : pouzit QListWidget alebo QTableWidget (pripadne nieco ine ako nejaky abstract viewmodel), narok na zakladne funkcionality : lahka rozsiritenlnost o dalsie ''stlpceky'' , moznost jednoduchej interakcie(doubleClick na polozku). Mat vsetky info na jednom okne / mat pop-up okna.
\section{Zobrazenie sémantického významu dát}
ako vyzobrazit semanticky vyznam roznych dat - descriptory, usb header, vyznam input dat roznych HID zariadeni
\section{Hexdump}
ako v qt urobit hexdump - do coho zobrazovat data(vytvorit si vlastny viewer dedeny od QAbstractScrollArea, pripadne niecoho ineho) vs najst nieco co uz v qt je a upravit to aby to sedelo poziadavkam. Vziat do uvahy bezne funkcie hexdumpu : selection mody(oznacit naraz hexa a im odpovedajuce printable), logicke oddelenie dat(napriklad farbami)









\include{content/kap05-vyvojova_dokumentacia}
\chapter{Možnosti rozšírenia}
Rozobrať čo všetko sa dá urobiť s tými dátami, ktoré už mám uložené v pamati, ale momentálne sa s nimi nič nedeje
\section{Ukladanie výstupu do súboru}
výstup analýzy do súboru(textového)
\section{Iná vizuálna reprezentácia dát}
Momentálne vyzobrazujem dáta prevažne v QTreeView alebo QTableView, ale vdaka tomu ako ich mám uložené + to že nad nimi operuje nejaký model ktorý vie vrátiť dáta na základe indexu, by nemuselo byť taká zložité pridať inú vizualizáciu dát(napríklad obrázkovú ako tu : https://www.usbmadesimple.co.uk/ums\_5.htm)
\section{Pridávanie nových interpreterov pre descripory}
pridanie nových druhov descriptorov - pridať nový interpreter do factory
\section{Pridanie intepreteru na interrupt tranfser}
pridanie analyzi interrupt transferu aj pre ine ako hid zariadenia
\subsection{Pridanie nových HID zariední}
nove HID zariadenie - pridanie do interrupt ''factory''
\section{Pridanie analýzy pre isochronous a bulk transfer}
semanticka analyza aj inych ako interrupt alebo control transferov - momentalne su rozpoznavane len v hexdumpe
\section{?Možnosť rozšírenia na iné platformy?}
uprava aplikacie aby bola prenositelna aj na ine platformy, co vsetko by tam bolo treba upravit(pravdepodobne nie vela, kedze qt je prenosne, a prakticky jedine co pouzivam spojene s windowsom su jeho structy na rozne descriptory)

\chapter{Užívateľská dokumentácia}
\label{udok:chap}

\section{Inštalácia}
nastavenie celkovej aplikácie, ale aj nainstalovanie USBPcap + wireshark a ich kombinácia pre live capture
\section{Orientácia v GUI aplikácie}
popis k jednotlivým tlačidlám gui
\section{Používanie aplikácie}
ako spustit live/offline capture, a celkovo ako pracovať s aplikáciou(popis funkcií - doubleClick na item => zobrazi sa pop-up okno s bližšou analýzou)




\include{content/kap08-zaver}

%%% Seznam použité literatury
\include{outro/literatura}

%%% Obrázky v bakalářské práci
%%% (pokud jich je malé množství, obvykle není třeba seznam uvádět)
\listoffigures

%%% Tabulky v bakalářské práci (opět nemusí být nutné uvádět)
%%% U matematických prací může být lepší přemístit seznam tabulek na začátek práce.
\listoftables

%%% Použité zkratky v bakalářské práci (opět nemusí být nutné uvádět)
%%% U matematických prací může být lepší přemístit seznam zkratek na začátek práce.
\chapwithtoc{Seznam použitých zkratek}

%%% Přílohy k bakalářské práci, existují-li. Každá příloha musí být alespoň jednou
%%% odkazována z vlastního textu práce. Přílohy se číslují.
%%%
%%% Do tištěné verze se spíše hodí přílohy, které lze číst a prohlížet (dodatečné
%%% tabulky a grafy, různé textové doplňky, ukázky výstupů z počítačových programů,
%%% apod.). Do elektronické verze se hodí přílohy, které budou spíše používány
%%% v elektronické podobě než čteny (zdrojové kódy programů, datové soubory,
%%% interaktivní grafy apod.). Elektronické přílohy se nahrávají do SISu a lze
%%% je také do práce vložit na CD/DVD. Povolené formáty souborů specifikuje
%%% opatření rektora č. 72/2017.
\appendix
%%% Přílohy k bakalářské práci, existují-li. Každá příloha musí být alespoň jednou
%%% odkazována z vlastního textu práce. Přílohy se číslují.
%%%
%%% Do tištěné verze se spíše hodí přílohy, které lze číst a prohlížet (dodatečné
%%% tabulky a grafy, různé textové doplňky, ukázky výstupů z počítačových programů,
%%% apod.). Do elektronické verze se hodí přílohy, které budou spíše používány
%%% v elektronické podobě než čteny (zdrojové kódy programů, datové soubory,
%%% interaktivní grafy apod.). Elektronické přílohy se nahrávají do SISu a lze
%%% je také do práce vložit na CD/DVD. Povolené formáty souborů specifikuje
%%% opatření rektora č. 72/2017.

\chapwithtoc{Prílohy}

\section{První příloha}

\openright
\end{document}
