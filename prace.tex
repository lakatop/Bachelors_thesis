%%% Hlavní soubor. Zde se definují základní parametry a odkazuje se na ostatní části. %%%

%% Verze pro jednostranný tisk:
% Okraje: levý 40mm, pravý 25mm, horní a dolní 25mm
% (ale pozor, LaTeX si sám přidává 1in)
\documentclass[12pt,a4paper]{report}
\setlength\textwidth{145mm}
\setlength\textheight{247mm}
\setlength\oddsidemargin{15mm}
\setlength\evensidemargin{15mm}
\setlength\topmargin{0mm}
\setlength\headsep{0mm}
\setlength\headheight{0mm}
% \openright zařídí, aby následující text začínal na pravé straně knihy
\let\openright=\clearpage

%% Pokud tiskneme oboustranně:
% \documentclass[12pt,a4paper,twoside,openright]{report}
% \setlength\textwidth{145mm}
% \setlength\textheight{247mm}
% \setlength\oddsidemargin{14.2mm}
% \setlength\evensidemargin{0mm}
% \setlength\topmargin{0mm}
% \setlength\headsep{0mm}
% \setlength\headheight{0mm}
% \let\openright=\cleardoublepage

%% Vytváříme PDF/A-2u
\usepackage[a-2u]{pdfx}

%% Přepneme na českou sazbu a fonty Latin Modern
\usepackage[slovak]{babel}
\usepackage{lmodern}
\usepackage[T1]{fontenc}
\usepackage{textcomp}

%% Použité kódování znaků: obvykle latin2, cp1250 nebo utf8:
\usepackage[utf8]{inputenc}

%%% Další užitečné balíčky (jsou součástí běžných distribucí LaTeXu)
\usepackage{amsmath}        % rozšíření pro sazbu matematiky
\usepackage{amsfonts}       % matematické fonty
\usepackage{amsthm}         % sazba vět, definic apod.
\usepackage{bbding}         % balíček s nejrůznějšími symboly
			    % (čtverečky, hvězdičky, tužtičky, nůžtičky, ...)
\usepackage{bm}             % tučné symboly (příkaz \bm)
\usepackage{graphicx}       % vkládání obrázků
\usepackage{fancyvrb}       % vylepšené prostředí pro strojové písmo
\usepackage{indentfirst}    % zavede odsazení 1. odstavce kapitoly
\usepackage{natbib}         % zajištuje možnost odkazovat na literaturu
			    % stylem AUTOR (ROK), resp. AUTOR [ČÍSLO]
\usepackage[nottoc]{tocbibind} % zajistí přidání seznamu literatury,
                            % obrázků a tabulek do obsahu
\usepackage{icomma}         % inteligetní čárka v matematickém módu
\usepackage{dcolumn}        % lepší zarovnání sloupců v tabulkách
\usepackage{booktabs}       % lepší vodorovné linky v tabulkách
\usepackage{paralist}       % lepší enumerate a itemize
\usepackage{xcolor}         % barevná sazba

% dodané mimo šablonu
\usepackage{graphicx}		% umisti obrazky vedle sebe
\usepackage{subcaption}		% umisti obrazky vedle sebe
\usepackage{enumitem}
\usepackage{dirtree}		% pro strom adresářů příloh
\usepackage{hyperref}
%\usepackage[cache=false]{minted}			% syntax highlight
%\usepackage{algpseudocode}  
%\usepackage{algorithm}
\usepackage{multicol}

\usepackage[shortcuts]{extdash}		% pro možnost zakázání dělení slov kolem pomlčky
									% např. pro slovo "chce-li", které lze zapsat "che\=/li"

%%% Údaje o práci

% Název práce v jazyce práce (přesně podle zadání)
\def\NazevPrace{Analyzátor USB paketov}

% Název práce v angličtině
\def\NazevPraceEN{USB Packet Analyzer}

% Jméno autora
\def\AutorPrace{Peter Lakatoš}

% Rok odevzdání
\def\RokOdevzdani{2021}

% Název katedry nebo ústavu, kde byla práce oficiálně zadána
% (dle Organizační struktury MFF UK, případně plný název pracoviště mimo MFF)
\def\Katedra{Katedra distribuovaných a~spolehlivých systémů}
\def\KatedraEN{Department of~Distributed and~Dependable Systems}

% Jedná se o katedru (department) nebo o ústav (institute)?
\def\TypPracoviste{Katedra}
\def\TypPracovisteEN{Department}

% Vedoucí práce: Jméno a příjmení s~tituly
\def\Vedouci{Mgr.~Pavel Ježek,~Ph.D.}

% Pracoviště vedoucího (opět dle Organizační struktury MFF)
\def\KatedraVedouciho{Katedra distribuovaných a~spolehlivých systémů}
\def\KatedraVedoucihoEN{Department of~Distributed and~Dependable Systems}

% Studijní program a obor
\def\StudijniProgram{Informatika}
\def\StudijniObor{Programování a~softwarové systémy}

% Nepovinné poděkování (vedoucímu práce, konzultantovi, tomu, kdo
% zapůjčil software, literaturu apod.)
\def\Podekovani{%
Týmto by som chcel poďakovať svojmu vedúcemu Mgr.~Pavlovi Ježekovi,~Ph.D., za jeho čas, ochotu, cenné rady a nápady bez ktorých by táto práca nemohla vzniknúť. Veľká vďaka patrí zároveň mojej rodine a priateľom za ich podporu počas celého môjho štúdia.
}

% Abstrakt (doporučený rozsah cca 80-200 slov; nejedná se o zadání práce)
\def\Abstrakt{%
USB zbernica je dnes jedným z najrozšírenejších spôsobov pripojenia periférií k~počítaču. 
Cieľom práce bolo vytvoriť software, ktorý analyzuje zachytenú komunikáciu medzi zariadnením pripojeným na~danú zbernicu a~počítačom.

Aplikácia prehľadným spôsobom vizuálne zobrazuje zanalyzované dáta -- konkrétne sa zameriava na~HID triedu zariadení a~ponúka aj sémantický význam jej úzkej podmnožiny do~ktorej patria myš, klávesnica a~joystick. Pri vizuálnej reprezentácii dát sa práca inšpiruje rôznymi dostupnými softwarmi, pričom rozlične kombinuje resp. dopĺňa ich vlastnosti a~implementuje z~nich tie, ktoré vníma ako najlepšie riešenie v danej situácii.

Dôležitá vlastnosť aplikácie je parsovanie HID Report Descriptoru vďaka ktorému bude v budúcnosti jednoduchšie pridať sémantickú analýzu rôznym ďalším HID zariadeniam. Celkový návrh aplikácie by mal ponúknuť možnosť budúcej implementácie ďalších USB tried pre~prípadné rozšírenie.
}
\def\AbstraktEN{%
The USB bus is the most common way of connecting peripherals to personal computers. The goal of this thesis is to create an application which analyzes communication between a device connected to this bus and a computer.

The application is capable to readably display analyzed data, with specific focus on HID class devices. The application implements semantic analysis of a subset of HID devices consisting of mice, keyboards and joysticks. The methods that the application uses to visually represent data are inspired by already existing applications, where our application combines them and impoves their capabilities to achieve better results.

Notable part of the application is its ability to parse HID Report Descriptor, to accomplish easier addition of new HID devices for semantic analysis. Overall design of the application is general enough to allow simple addition of analysis for other USB classes.
}

% 3 až 5 klíčových slov (doporučeno), každé uzavřeno ve složených závorkách
\def\KlicovaSlova{%
{USB} {HID} {Analyzátor}
}
\def\KlicovaSlovaEN{%
{USB} {HID} {Analyzer}
}

%% Balíček hyperref, kterým jdou vyrábět klikací odkazy v PDF,
%% ale hlavně ho používáme k uložení metadat do PDF (včetně obsahu).
%% Většinu nastavítek přednastaví balíček pdfx.
\hypersetup{unicode}
\hypersetup{breaklinks=true}

%% Definice různých užitečných maker (viz popis uvnitř souboru)
\include{makra}

%% Titulní strana a různé povinné informační strany
\begin{document}
\include{intro/titulka}

%%% Strana s automaticky generovaným obsahem bakalářské práce

\tableofcontents

%%% Jednotlivé kapitoly práce jsou pro přehlednost uloženy v samostatných souborech
\chapter{Úvod}

Možnosti pripojenia rôznych periférií k zariadeniu sú v dnešnej dobe rozsiahle. Aj napriek tomu, že technológia každým dňom napreduje a svet sa uberá viac bezdrôtovým smerom, je USB stále najrozšírenejší sériový spôsob prenášania dát. Už z názvu \uv{Universal Serial Bus} je jasné, že rozpätie zariadení, ktoré možno k tejto zbernici pripojiť je obrovské. Práve preto je USB protokol jeden z najkomplexnejších protokolov ktoré sa využívajú na komunikáciu.

V tejto práci sa pozrieme na užšiu podmnožinu USB protokolu a na komunikáciu s dopredu vybratými perifériami, konkrétne myšou, klávesnicou a joystickom.

\section{USB}
vysvetlenie zakladnych pojmov spojenych USB: historia, usb port/conector, plug and play(https://docs.microsoft.com/en-us/windows-hardware/drivers/kernel/introduction-to-plug-and-play), host-master, low/full/high speed zariadenia

\section{Existujúce aplikácie}

Momentálne existuje niekoľko známych aplikácií ktoré slúžia na analýzu USB paketov. V tejto kapitole si ich zopár ukážeme, pričom mnohé z nich nám poslúžili ako inšpirácia pri ppísaní našej práce a riešení konkrétnych problémov na ktoré sa pozrieme v nasledujúcich kapitolách. 

Je nutné upozorniť, že väčšina dnešných analyzátorov sú platené aplikácie, prípadne majú odomknuté len základné vlastnosti s možnosťou dokúpenia si plnej verzie. Práve preto pri ich prípadnom porovnávaní budeme brať do úvahy len funkcionalitu, ktorá je dostupná zadarmo.

\subsection*{Wireshark}

Pravdepodobne najznámejšia third-party aplikácia na analýzu paketov. Jeho funkcionalita je veľmi rozsiahla, a vzhľadom na to, že sa jedná o open-source projekt, neustále rastie. Zameriava sa hlavne na analýzu sieťových paketov. Napriek tomu podporuje spoluprácu s rôznymi inými sniffermi. Jeden z takýchto snifferov je USBPcap ktorý zachytáva USB komunikáciu, a tak nie je prekvapivé, že Wireshark podporuje analýzu paketov aj nad touto zbernicou. 

Vzhľadom na obľúbenosť a rozsiahlosť programu, nám Wireshark poslúžil ako referenčná aplikácia, z ktorej sme čerpali celkovú inšpiráciu na funkcie, ktoré by mal bežný analyzátor paketov spĺňať. Medzi tie úplne základné patrí napríklad hexdump dát nad ktorými prebieha analýza, ale napríklad aj spôsob vyobrazenia rôznych deskriptorov, ktoré je riešené cez stromovú štruktúru. Medzi viac špecifické funkcie ktoré sme neskôr implementovali aj v našom programe patrí napríklad detailnejšie vyobrazenie jednotlivých bytov a ich význam, ako je možné vidieť nižšie na obrázku~\ref{obr:uvod:byte_detail_foto}. Túto vlastnosť aj napriek jej využitiu mnohé konkurenčné aplikácie postrádajú.

Jeho výhoda je hlavne v tom, že podporuje širokú škálu deskriptorov a plná verzia programu je dostupná úplne zadarmo. Z pohľadu užívateľa je až prekvapivé, že aj napriek rozsiahlosti programu je aplikácia veľmi user-friendly orientovaná a dopĺňa ju veľmi intuitívne užívateľské rozhranie.

\begin{figure}
	\centering
	\includegraphics[width=\textwidth]{img/uvod_byte_detail}
	\caption{Ukážka vyobrazenia jednotlivých bytov.}
	\label{obr:uvod:byte_detail_foto}
\end{figure}

\subsection*{Device Monitoring Studio}

Plná verzia aplikácie je platená. Verzia zadarmo ponúka analýzu sieťových a USB paketov, tak ako aj analýzu komunikácie prebiehajúcej cez sériový port. 

Ako prvé na aplikácii zaujme spôsob zvolenia si zariadenia s ktorým bude sledovaná komunikácia. Je implementovaný štýlom stromovej štruktúry ako je ukázané na obrázku~\ref{obr:uvod:treeview_foto} nižšie. Už základná verzia programu poskytuje veľmi rosiahlu funkcionalitu. Používateľovi je umožnené posielať zariadeniu požiadavky definované v USB špecifikácii(ODKAZ). Aplikácia taktiež zobrazuje sémantickú analýzu niektorých HID zariadení, ale nie v práve najlepšej podobe. 

Na základe výstupu sa dá povedať, že sémantická analýza je naimplementovaná skôr obecne a pri niektorých položkách je namiesto ich významu napísané \uv{Unknown}. Taktiež nie je veľmi jasné odkiaľ sa dané hodnoty berú, keďže k nim chýba ich dátová reprezentácia. Medzi zachytenými paketami sa prvotne nezobrazujú tie, ktoré označujú nakonfigurovanie daného zariadenia (je nutné ho odpojiť a znova napojiť počas monitorovania). Užívateľské rozhranie je veľmi chaotické a chvíľu trvá, kým človek nájde čo i len základné informácie ako napríklad hlavičky ku jednotlivým paketom.  Nepoteší ani fakt, že verzia zadarmo nedovoľuje monitorovanie dlhšie ako 10 minút a maximálny počet monitorovaní za jeden deň je taktiež 10.

\newpage

\begin{figure}[!htb]
	\centering
	\includegraphics[width=\textwidth]{img/uvod_treeview}
	\caption{Ukážka stromovej štruktúry na zvolenie si zariadenia s ktorým bude zachytávaná komunikácia.}
	\label{obr:uvod:treeview_foto}
\end{figure}

\section{Požadované funkcie}
Na základe minulých príkladov už existujúcich aplikácií sme si mohli všimnúť, že všetky obsahujú niekoľko základných funkcií, ktoré by mal obsahovať každý analyzátor. V tejto sekcii zhrnieme funkcie, ktoré sme si zvolili pre našu aplikáciu. Zo základných funkcií sme si vybrali tie, ktoré považujeme za absolútne nevyhnutné pre každý analyzátor. Následne sme sa niekoľko z nich pokúsili akýmsi spôsobom vylepšiť tak, aby maximalizovali ich využiteľnosť a zároveň čo najlepšie zlepšovali prácu s aplikáciou jej užívateľovi. Zároveň ich dopĺňame o pokročilejšie funkcie z ktorých sú niektoré viac špecificky zamerané.

\subsubsection*{Základné}

Výsledná aplikácia by teda mala splňovať nasledujúce zákadné požiadavky:

\begin{enumerate}[label=\textbf{P\arabic*}]
	\item Mala by byť schopná analyzovať USB pakety zachytené v \textbf{pcap} formáte pomocou \textbf{USBPcap} snifferu.
	\item Mala by podporovať sémantickú analýzu pre všetky základné USB descriptory spomenuté v USB 2.0 dokumentácii(ODKAZ kapitola 9.6) (ako napríklad \textit{Device descriptor}, \textit{Interface descriptor}, \textit{Endpoint descriptor},...)
	\item Mala by vedieť rozumne zobraziť dáta ktoré \textbf{USBPcap} zachytí a uloží. Pod pojmom rozumne myslíme spôsob zobrazenia pomocou hexdumpu.
	\item Mala by na prvý pohľad jasne zobraziť základné informácie o každom analyzovanom pakete (ako napr. dĺžka paketu, typ prenosu, ...) a pri bližšom skúmaní jednotlivých paketov detailnejšie zobraziť celú jeho hlavičku.
	\item Detailnejšie informácie o pakete budú zobrazované na základe interakcie užívateľa s aplikáciou.
\end{enumerate}

\subsubsection*{Pokročilejšie}

Zároveň sme aplikáciu doplnili o niekoľko pokročilejších a špecificky zameraných požiadaviek:

\begin{enumerate}[label=\textbf{P\arabic*},resume]
	\item Mala by určitým spôsobom uľahčiť používateľovi orientáciu v hexdumpe.
	\item Mala by byť schopná rozparsovať \textbf{HID Report Descriptor} takým štýlom, aby bolo neskôr možné sématnicky reprezentovať input určitých HID zariadení
	\item Mala by byť schopná vhodným spôsobom vizuálne zobraziť sémantický význam dát posielaných danou podmnožinou HID zariadení do ktorej patrí myš, klávesnica a joystick.
	\item V miestach kde to dáva zmysel, by aplikácia mala byť schopná zobrazovať význam dát až na úrovni jednotlivých bitov.
\end{enumerate}

\section{Ciele práce}
vytvorit aplikaciu
navrh musi byt dostatocne obecny aby sa dal rozsirit o dalsie USB protokoly
dostatocne obecny navrh pre lahke pridavanie novych HID zariadeni
prehladny interface








\chapter{USB}
V tejto kapitole si rozšírime znalosti fungovania USB, ktoré sme získlai v sekcii~\ref{uvod:sec:zakl_pojmy}. Najprv si vysvetlíme ako prebieha komunikácia medzi USB zariadením a hostom, následne sa pozrieme na konfiguráciu USB zariadenia po pripojení na zbernicu, prejdeme si niektoré základné USB descriptory a ako posledné si ešte vysvetlíme základnú stavbu USB na Windowse.

\section{Komunikácia}
Detailný popis komunikácie a presunu dát po zbernici je opísaný v USB 2.0 špecifikácii v kapitole 5 (ODKAZ). My sa momentálne zameriame len na určité časti, ktoré sú potrebné pre našu prácu. Každé USB zariadenie v sebe obsahuje tzv. \textit{endpointy}(ODKAZ) -- môžeme to považovať za akúsi koncovku komunikácie medzi hostom a zariadením. Z technického hľadiska sa jedná o hardware, ktorý je schopný v sebe uchovať dáta (memory buffer). Ako už vieme z kapitoly~\ref{uvod:sec:zakl_pojmy}, komunikáciu riadi USB host, a tá prebieha práve pomocou týchto endpointov -- v prípade ak chce host poslať určité dáta zariadeniu, zapíše ich do jeho konkrétneho endpointu. V prípade ak chce zariadenie poslať určité dáta hostovi, zapíše si ich do daného endpointu, odkiaľ ich host potom prečíta.

\subsection*{Endpoint}
USB zariadenie typicky pozostáva z niekoľkých na sebe nezávislých endpointov. Každý endpoint je potom jednoznačne určený:
\begin{enumerate}
\item Adresou USB zariadenia -- tá je pridelená USB zariadeniu pri jeho konfigurácii v momente pripojenia na zbernicu.
\item Číslom endpointu -- unikátne číslo, ktoré určuje výrobca zariadenia.
\item Smerom prenosu dát -- host $\longrightarrow$ device alebo device $\longrightarrow$ host.
\end{enumerate}

Do momentu pokiaľ neprebehne konfigurácia USB zariadenia a jeho endpointov, sú endpointy s adresou inou ako 0 v neurčitom stave a nemusia byť pre hosta dostupné. Endpoint s adresou 0, inak nazývaný aj ako \uv{default endpoint} alebo \uv{Endpoint0}, slúži na nakonfiguranie daného USB zariadenia. Výrobca zariadenia je povinný poskytnúť aspoň 1 Endpoint0 pre každý smer pohybu dát, prípadne 1 Endpoint0 s možnosťou prenosu oboma smermi. Funkcie (USB zariadenia) môžu obsahovať aj ďalšie endpointy s adresou inou ako 0. Tieto endpointy slúžia na prenos dát špecifických pre dané zariadenie (napríklad na posielane inputu myši). Rôzne endpointy môžeme zlučovať do určitých množín podľa ich funkcionality. Takúto množinu endpointov potom nazývame \textit{interface}. Niektoré USB zariadenia potom môžu pozostávať z viacerých interfacov, ktoré budú reprezentovať rozličné USB triedy.

\subsection*{Pipe}
USB pipe je termín označujúci spojenie medzi konkrétnym endpointom USB zariadenia a Host Controllerom (interface medzi hostom a zbernicou). Reprezentuje schopnosť prenášať dáta medzi hostom a endpointom pomocou memory bufferu. Pipe ktorá pozostáva z dvoch Endpoint0 sa nazýva \uv{Default Control Pipe} a je prístupná v momente pripojenia zariadenia na zbernicu a slúži na konfiguráciu daného zariadenia (po konfigurácii môže mať aj iné využitie, ktoré špecifikuje samotný výrobca). Ostatné pipy s ďalšími endpointami (za predpokladu, že existujú) sú vytvorené až po konfigurácii zariadenia. 



\section{Konfigurácia}
Konfigurácia USB zariadenia je detailne opísaná v USB 2.0 špecifikácii (ODKAZ 9.2.3). Predtým ako začneme využívať funkcionalitu pripojeného USB zariadenia, je USB host zodpovedný za jeho nakonfigurovanie. Počas konfigurácie posiela USB host zariadeniu tzv. \uv{Device Requesty}, na ktoré dané zariadenie odpovedá cez Default Control Pipe. Tieto requesty sú špecifikované v \textit{Setup Pakete} -- štruktúra veľká 8 bytov so štandardným formátom definovaným v USB špecifikácii 2.0 (ODKAZ 9.3). Existuje niekoľko základných requestov (ODKAZ 9.4), na ktoré musí každe USB zariadenie vedieť reagovať. Patria medzi ne napríklad:
\begin{itemize}
\item Get Descriptor -- vypýta si od zariadenia konkrétny descriptor, ktorý mu zariadenie pošle ako odpoveď na tento request (za predpokaldu, že daný descriptor existuje)
\item Set Configuration -- nastaví konkrétnu konfiguráciu zariadeniu
\end{itemize}

Bežný postup konfigurácie je, že si USB host vypýta rôzne descriptory od zariadenia, ktoré určujú jeho schopnosti (napr. \textit{Configuration Descriptor}, \textit{Interface Descriptor}, \textit{Endpoint Desriptor}) a potom pomocou requestu Set Configuration nastaví požadovanú konfiguráciu (a ak je to nutné, zvolí rôzne dodatočné nastavenie interfacov).



\section{USB Descriptory}
Teraz si prejdeme niekoľko základných USB descriptorov a ich jednotlivé položky, pretože ich význam budeme potrebovať neskôr v tejto práci. 


\subsection*{Configuration Descriptor}
Configuration Descriptor (ODKAZ 9.6.3) opisuje informácie o konkrétnej konfigurácii USB zariadenia. Obsahuje položku \textit{bConfigurationValue} -- číslo reprezentujúce konkrétnu konfiguráciu, ktorú USB host použije ako parameter v \texttt{SetConfiguration()} requeste v prípade, že chce nastaviť práve túto konfiguráciu. Každé zariadenie má aspoň jeden Configuration Descriptor. Každá konfigurácia obsahuje aspoň jeden interface a každý interface má nula alebo viac endpointov. V prípade, že si host vyžiada od zariadenia Configuration Descriptor, dostane spolu s ním aj všetky súvisiace Interface a Endpoint descriptory.


\subsection*{Interface Descriptor}
Interface Descriptor (ODKAZ 9.6.5) opisuje šepcifický interface konkrétnej konfigurácie USB zariadenia. Ak zariadenie podporuje viac ako jeden interface, tak všetky Interface Descriptory spolu s im odpovedajúcimi Endpoint Descriptormi sú vrátené ako odpoveď na \texttt{GetConfiguration()} request (k Interface Descriptoru nie je možný priamy prístup pomocou \texttt{GetDescriptor()} alebo \texttt{SetDescriptor()} requestom). Ak interface používa len Endpoint0, tak za Interface Descriptorom nenasleduje žiaden Endpoint Descriptor. Endpoint0 takisto nie je započítaný v položke Interface Descriptoru \textit{bNumEndpoints}, ktorá udáva počet endpointov konkrétneho interfacu.


\subsection*{Endpoint Descriptor}
Endpoint Descriptor (ODKAZ 9.6.6) poskytuje hostovi informácie o konkrétnom endpointe. Tento descriptor obsahuje aj informácie na základe ktorých je host schopný určiť bandwidth konkrétneho endpointu -- množstvo dát prenesených za jednotku času (typicky bity za sekundu = b/s, alebo byty za sekundu = B/s). Takisto ako aj  pri Interface Descriptore, nie je možné k nemu priamo pristupovať pomocou \texttt{GetDescriptor()} alebo \texttt{SetDescriptor()} requestov, ale je súčasťou odpovede na \texttt{GetConfiguration()} request.



\section{Windows}
Keďže Windows je hlavná platforma na ktorú mierime s našou aplikáciou, priblížime si ako sú reprezentované jednotlivé USB zariadenia a priebeh komunikácie na danej zbernici.







\chapter{Analýza}
\section{Získanie USB packetov}
Na získavanie USB paketov nám bude obecne slúžiť paket sniffer. Väčšina paket analyzátorov má implementované vlastné sniffery a preto sme sa o to pokúsili tiež. Narazili sme ale na niekoľko zásadných problémov, ktoré sa úzko viažu s platformou na ktorú cielime s našou aplikáciou~--~Windows.

Microsoft dokumentácia podrobnejšie opisuje komunikáciu medzi HID zaridením a kernel/user-mode aplikáciou~\cite{hid_opening_collections}. Keďže naša aplikácia beží v user-mode, prejdeme si tento spôsob:
\begin{enumerate}
\item Aplikácia nájde a identifikuje HID zariadenie.
\item Aplikácia pomocou metódy \textit{CreateFile} otvorí spojenie s HID zariadením.
\item Aplikácia pomocou \textit{HID API}~\cite{hid_api} metód \textit{HidD\_Xxx} získa \textit{Preparsed Data} a informácie ohľadom HID zariadenia.
\item \textbf{Aplikácia použije metódu \textit{ReadFile} resp. \textit{WriteFile} na získanie inputu zariadenia resp. poslanie reportu zariadeniu.}
\item Aplikácia pomocou \textit{HID API}~\cite{hid_api} metód \textit{HidP\_Xxx} interpretuje HID reporty.
\end{enumerate}

\subsection{Windows exclusive mód}
Windows má definovaný tzv. \textit{Access Mode}, ktorý určuje restrikciu prístupu \textit{HID Clienta} k HID zariadeniu. 
Ten môže byť buď \textit{Shared} alebo \textit{Exclusive}. \textit{Exclusive Mode} zabraňuje ostatným \textit{HID Clientom} v zachytávaní alebo získavaní inputu HID zariadenia, pokiaľ nie sú hlavným príjemcom daného inputu. Preto z bezpečnostných dôvodov otvára \textit{RIM (Raw Input Manager)} niektoré zariadenia v \textit{Exclusive Mode}.

Ak je zariadenie otvorené v \textit{Exclusive Mode}, aplikácia má stále prístup k niektorým jeho údajom pomocou  \textit{HID API}~\cite{hid_api} metód  \textit{HidD\_\textbf{Get}Xxx}. Tieto metódy nám obecne umožnia získať niektoré descriptory zariadenia, tak ako aj jeho \textit{Preparsed Data}. Nie je nám ale umožnené volať metódu \textit{ReadFile}, takže nemáme akým spôsobom zachytávať komunikáciu HID zariadenia s clientom.

Tabuľka zariadení~\cite{hid_access} (obrázok~\ref{obr:kap3:access_mode}), ktoré \textit{RIM} otvára v \textit{Exclusive Mode} obsahuje aj tie, ktoré sme si v úvode zvolili ako podmnožinu HID zariadení na analýzu -- myš a klávesnica.

\begin{figure}[!htb]
	\centering
	\includegraphics[width=\textwidth]{img/kap3_access_mode}
	\caption{Tabuľka zariadenía ich \textit{Access Mode}. Zariadenia postupne po riadkoch -- myš, joystick a klávesnica}
	\label{obr:kap3:access_mode}
\end{figure}

\newpage

\subsection{Známe knižnice}
opisat zakladne kniznice na sledovanie USB zbernice a preco som ich nemohol pouzit : libUSB, hidAPI

\subsection{Driver}
TU povedat riesenie - pouzitie driveru na komunikaciu so zariadenim. Existujuce windows drivery -- moufiltr, Kbdfiltr - nefunguju pre USB

TU spomenut posledne mozne riesenie - napisanie vlastneho filter driveru. 

\subsection{Third-party aplikácie}
opisat odkial nakoniec ziskavam packety - USBPcap a Wireshark









\section{Spracovávanie pcap súborov}
moznosti ako citat pcap subory : bud pouzit uz existujucu kniznicu : na linuxe Libpcap, windows NPcap(deprecated WinPcap), alebo citat subory manualne : std::istream alebo QFile
\section{Sémantická reprezentácia dát}
ako si z dat vytiahnut udaje ktore su potom pouzite na semanticku analyzu implementovanych HID zariadeni : HID Report parser, InputValues a EndpointDevice struct.
Nasledne sparovanie - ako vybrat spravny report pre konkretny input
\section{Voľba frameworku}
obecne co by som od toho GUI priblizne chcel, potom opisat preco som si vybral prave Qt a v nasledujucich kapitolach opisat rozhodnutia uz v Qt
dovod preco som si zvolil qt namiesto inych c++ GUI frameworkov(napriklad sfml)
\section{Zobrazenie základných informácií}
ako zobrazovat zakladne info o packete : pouzit QListWidget alebo QTableWidget (pripadne nieco ine ako nejaky abstract viewmodel), narok na zakladne funkcionality : lahka rozsiritenlnost o dalsie ''stlpceky'' , moznost jednoduchej interakcie(doubleClick na polozku). Mat vsetky info na jednom okne / mat pop-up okna.
\section{Zobrazenie sémantického významu dát}
ako vyzobrazit semanticky vyznam roznych dat - descriptory, usb header, vyznam input dat roznych HID zariadeni
\section{Hexdump}
ako v qt urobit hexdump - do coho zobrazovat data(vytvorit si vlastny viewer dedeny od QAbstractScrollArea, pripadne niecoho ineho) vs najst nieco co uz v qt je a upravit to aby to sedelo poziadavkam. Vziat do uvahy bezne funkcie hexdumpu : selection mody(oznacit naraz hexa a im odpovedajuce printable), logicke oddelenie dat(napriklad farbami)









\chapter{Vývojová dokumentácia}
\label{chap:vyvoj_dok}
\section{Architekrúra aplikácie}
\section{Jadro aplikácie}
\subsection{USB\_Packet\_Analyzer}
riadi celkovy beh programu, reaguje na input od uzivatela
\subsection{Item Manager}
spracovanie samostatneho packetu a ulozenie dat o nom
\subsection{DataViewer}
trieda ktora ma na starosti vyskakovacie okno po dvojkliku a item a nasledne reaguje na input od uzivatela v okne
\subsection{TreeItem}
reprezentuje jednotlive nody v stromovej strukture ktora sa potom vyuziva na zobrazenie dat v QTreeView
\section{Modely}
\subsection{AdditionaldataModel}
model na spravovanie zvysnych dat(data ktore nie su sucastou hlavicky packetu)
\subsection{ColorMapModel}
vyobrazenie pomocnej mapy na lepsie sa zorientovanie v zvyraznemom hexdumpe
\subsection{DataViewerModel}
model na hexdump - prenasa hex/printable a zaroven o co vlastne ide(konkretny descriptor, interrupt data, ...)
\subsection{TreeItemBaseModel}
model na QTreeView ktorz vyuziva TreeItem
\subsection{USBPcapHeaderModel}
model na QTreeView ale specialne pre USBPcap hlavicku packetu
\section{Interpretery}
\subsection{BaseInterpreter}
abstractna trieda od ktorej dedia vsetkz interpretery
\subsection{Interpreter factory}
facory trieda na pridelenie konkretneho interpreteru za runtimu kvoli jednoduchosti na lepsie rozsirenie programu do buducnosti
\subsection{Interpretery descriptorov}
Config,Device,Setup,String,...
\subsection{Interrupt transfer interpretery}
obecne interrupt transfer interpreter - sluzi skor ako factory na rozne doteraz implementovane HID zariadenia
\subsubsection{Joystick interpreter}
\subsubsection{Mouse interpreter}
\subsubsection{Keyboard interpreter}
\section{Delegáti}
\subsubsection{DataViewerDelegate}
Qt delegat - stara sa o highlight hexdumpu
\section{HID}
\subsection{HIDDevices}
staticka trieda, drzi vsetky rozpoznane HID zariadenia a obsahuje funkcie specificke nich - parsovanie HID Report descriptoru
\section{Práca so súbormi}
\subsection{FileReader}
praca zo suborom a predavanie precitanych dat, offline/online capture, QFile vs std::istream
\section{Globálne dáta}
\subsection{ConstDataHolder}
staticka trieda na drzanie si konstant ktore su potrebne napriec celym programom. Mapovanie z enumu do jeho stringovej reprezentacie
\subsection{PacketExternStructs}
obsahuje definiciu vsetkych dolezitych USBPcap structov, pcap structov, enumov a vsetkych structov ktore pouzivam v aplikacii












\chapter{Možnosti rozšírenia}
V tejto kapitole sa pozrieme na rôzne možnosti a smery, akými by sme mohli našu aplikáciu ďalej rozšíriť.
\section{Ukladanie výstupu do súboru}
Niektoré analyzátory ponúkajú ukladanie výpisu do rôznych súborov. V prípade ak by sme sa rozhodli našu aplikáciu rozšíriť o možnosť ukladania analýzy do súboru, museli by sme si rozmyslieť niekoľko nasledujúcich vecí:

\paragraph{Formát výpisu}
\hfill \break
Akým spôsobom bude vyzerať výstup analýzy. Na výber máme viacero možností od obyčajného textového výstupu až po obrázky alebo rôzne tabuľky. Na začiatok by nebolo príliš zložité zaintegrovať výpis analýzy v textovom formáte. Všetky dáta máme interne reprezentované v podobe QByteArray alebo stromovej štruktúry, ktoré je možné jednoducho prechádzať. Samozrejme musíme myslieť na to, že textový výpis nie je tak flexibilný a interaktívny ako prípadný QTreeView, takže vyobrazenie stromovej štruktúry by pravdepodobne nevyzeralo ideálne.

\paragraph{Rozhodnutie užívateľa, či bude danú analýzu ukladať do súboru alebo nie}
\hfill \break
Musíme myslieť na to, v akej fáze programu umožníme užívateľovi sa najneskôr rozhodnúť o ukladaní danej analýzy do súboru. Momentálne si všetky dáta uchovávame v samostatných QTableWidgetItemoch uložených v QTableWidgete hlavného okna aplikácie. Užívateľ má ale možnosť toto okno \uv{vyčistiť} pomocou tlačidla Clear~\ref{kap04:sec:clear_button}, ktoré vymaže všetky QTableWidgetItemy a tým pádom stratíme referenciu na všetky dáta, ktoré v sebe dané itemy uchovávajú. Doteraz sme sa tým nemuseli zaoberať, pretože dané dáta sem potrebovali na analýzu paketov reprezentovaných riadkom v QTableWidgete, takže akonáhle bol riadok vymazaný pomocu Clear tlačidla, detailnejšia analýza daného paketu už nebola možná. Niektoré možné riešenia tohto problému by mohli byť nasledujúce:
\begin{itemize}
\item V prípade, že umožníme užívateľovi si uložiť analýzu do súboru len do momentu pokiaľ nestlačí tlačidlo Start, máme na výber prakticky 2 možnosti:
\begin{enumerate}
\item \label{kap05:sec:refer} Ukladať si referencie na jednotlivé QTableWidgetItemy až do momentu skončenia programu, kedy jednorázovo zapíšeme do súboru celú analýzu.
\item \label{kap05:sec:priebez} Priebežne zapisovať do súboru analýzu jednotlivých paketov.
\end{enumerate}
Riešenie~\ref{kap05:sec:refer} prináša tú nevýhodu, že si musíme držať v pamäti všetky dáta paketov a zároveň by ukončenie aplikácie trvalo dlhšiu dobu, pretože by sa musela postupne vykonať a zapísať detailná analýza každého paketu. Naopak riešenie~\ref{kap05:sec:priebez} nás zbavuje oboch týchto problémov, ale mohla by mať nepriaznivý dopad na užívateľskú interakciu s aplikáciou, pretože v pozadí by prebiehal proces zapisovania a analýzy. To by sa samozrejme dalo optimalizovať rôznymi spôsobmi -- zapisovať len v momente pokiaľ užívateľ neinteraguje s aplikáciou, využiť metódy paralelného programovania a na zápis/analýzu použiť viac vláken, atď.

\item V prípade, že umožníme užívateľovi rozhodnúť o zápise do súboru hocikedy v priebehu používania aplikácie, musíme mať dáta jednotlivých paketov k dispozícii aj po ich odstráneni z QTableWidgetu tlačidlom Clear. Ako toho dosiahnuť sme naznačili vyššie v riešení~\ref{kap05:sec:refer} spolu s jeho nevýhodami.
\end{itemize}

Dôležité je ale spomenúť, že pridaním ukladania výpisu do súboru nijakým vážnym spôsobom nezasahujeme do programu a nemodifikujeme už naimplementované časti. Všetky dáta máme v aplikácii pripravené a jediné čo nám ostáva vyriešiť je ich spracovanie do súboru.

\section{Iná vizuálna reprezentácia dát}
Momentálne vyobrazujeme dáta za pomoci viewerov -- QTableView, QTreeView. Ako sme si už spomínali vyššie, celé to prebieha na základe Model/View architektúry, ktorá nám umožnuje od seba oddeliť dáta a spôsob akým ich vyobrazujeme. Na základe toho by pre nás nemalo byť tak náročné implementovať nové spôsoby vizuálnej reprezentácie dát. To by sme vedeli dosiahnuť pridaním nového typu vieweru. Mohli by sme si vybrať z už existujúcich, alebo si kľudne naimplementovať vlastný, ktorý by odpovedal našim predstavám vyobrazenia dát. Momentálne máme vytvorených viacero modelov pre špecifické časti našich dát (HexdumpModel pre všetky dáta paketu na vyobrazenie hexdumpu, USBPCapHeaderModel pre hlavičku paketu na jej vyobrazenie pomocou stromovej štruktúry, atď.). Všetky tieto modely sme implementovali z dôvodu aby sme jednoducho vedeli vyobraziť dáta v jednotlivých vieweroch. Preto by pridanie nového vieweru malo pravdepodobne za následok nutnosť implementovať aj tomu odpovedajúci model.

Zaujímavý spôsob vyobrazenia dát by bolo napríklad pomocou koláčového grafu. Vedeli by sme tak vizuálne zobraziť pomer rôznych dát ako napríklad:
\begin{itemize}
\item pomer rôznych typov prenosov (Control/Interrupt/Bulk/Isochronous) počas zachytávania paketov.
\item pomer veľkosti dát hlavičky paketu a zvyškových dát.
\item pomer veľkosti dát poslaných zariadením a USB hostom.
\item pomer zvyškových dát v paketoch vzhľadom na typ prenosu. 
\end{itemize}

Takisto by mohlo byť zaujímavé vyobraziť dáta viac grafickým spôsobom, napríklad ako je ukázané na obrázku~\ref{obr:kap5:graphics_packets} nižšie.

\begin{figure}[!htb]
	\centering
	\includegraphics[width=12cm]{img/kap05_graphics_packets}
	\caption{Ukážka grafického vyobrazenia paketov. Obrázok prevzatý zo stránky USB Made Simple~\cite{usbmadesimple_graphics}}
	\label{obr:kap5:graphics_packets}
\end{figure}

\newpage
\section{Pridávanie nových interpreterov pre descriptory}
pridanie nových druhov descriptorov - pridať nový interpreter do factory
\section{Pridanie intepreteru na interrupt tranfser}
pridanie analyzy interrupt transferu aj pre ine ako hid zariadenia
\subsection{Pridanie nových HID zariední}
nove HID zariadenie - pridanie do interrupt ''factory''
\section{Pridanie analýzy pre isochronous a bulk transfer}
semanticka analyza aj inych ako interrupt alebo control transferov - momentalne su rozpoznavane len v hexdumpe
\section{?Možnosť rozšírenia na iné platformy?}
uprava aplikacie aby bola prenositelna aj na ine platformy, co vsetko by tam bolo treba upravit(pravdepodobne nie vela, kedze qt je prenosne, a prakticky jedine co pouzivam spojene s windowsom su jeho structy na rozne descriptory)

\chapter{Užívateľská dokumentácia}
\label{udok:chap}

V tejto kapitole si ukážeme možnosti používania našej aplikácie, tak ako aj popis jej samotného užívateľského rozhrania. Predtým si ale ešte musíme nainštalovať niektoré aplikácie potrebné k jej plnému využitiu.

\section{Inštalácia}
Našu aplikáciu nie je potrebné nijakým spôsobom inštalovať. Stačí otvoriť súbor \textit{USB\_Packet\_Analyzer.exe} v priečinku \textit{Release}, ktorý je súčasťou prílohy tejto práce. Podporovaná platforma našej aplikácie je momentálne iba Windows 10 64-bit verzie 20H2. Pre vytváranie súborov na analýzu si ale budeme musieť nainštalovať aplikácie USBPcap a Wireshark. V prípade, že nemáme záujem o analýzu paketov v reálnom čase, bude nám stačiť aplikácia USBPcap.

\subsection{USBPCap}
Aplikáciu USBPcap nainštalujeme podľa nasledujúcich krokov:
\begin{enumerate}
\item Prejdeme na stránku \uv{https://desowin.org/usbpcap/}.
\item V \textit{Download} časti klikneme na \uv{USBPcapSetup-1.5.4.0.exe}\footnote{Číslo sa môže líšť na základe aktuálnej verzie USBPCapu.}, čím začne sťahovanie USBPCap setupu.
\item Spustíme USBPCapSetup, ktorý sme stiahli v predchádzajúcom kroku, čo nám otvorí dialógové okno inštalácie.
\item Akceptujeme licenčné podmienky a klikneme na tlačidlo \uv{Next}. Opakujeme aj na ďalšej strane okna.
\item Necháme zaškrtnuté všetky komponenty a klikneme na \uv{Next}.
\item Zvolíme si priečinok do ktorého chceme USBPCap nainštalovať a klikneme na \uv{Install}.
\item Po inštalácii klikneme na \uv{Close}, čím inštalácia končí. Následne musíme ešte reštartovať počítač.
\end{enumerate}

\subsection{Wireshark}
Pre nainštalovanie Wiresharku budeme postupovať podľa nasledujúcich krokov:
\begin{enumerate}
\item Prejdeme na stránku \uv{https://www.wireshark.org/\#download}.
\item V časti \textit{Stabe Release (3.4.5)\footnote{Číslo sa môže líšť na základe aktuálnej verzie Wiresharku.}} klikneme na položku \uv{Windows Installer (64-bit)} a stiahneme inštalačný súbor Wiresharku.
\item Spustíme súbor, ktorý sme si stiahli v minulom kroku, čím sa nám otvorí dialógové okno inštalácie.
\item Klikneme na tlačidlo \uv{Next} a následne na tlačidlo \uv{Noted}.
\item Necháme zaškrtnuté všetky možnosti a klikneme na tlačidlo \uv{Next}.
\item V časti \uv{Create Shortcuts} zaškrtneme možnosti podľa osobných preferencií a klikneme na tlačidlo \uv{Next}.
\item Zvolíme si priečinok do ktorého chceme Wireshark nainštalovať a klikneme na tlačidlo \uv{Next}.
\item V prípade, že nemáme nainštalovaý Npcap, zaškrtneme možnosť na jeho nainštalovania a klikneme na tlačidlo \uv{Next}.
\item USBPCap by sme už mali mať nainštalovaný podľa predchádzajúceho návodu, takže len klikneme na tlačidlo \uv{Install}.
\item Po dokončení inštalácie klikneme na tlačidlo \uv{Next} a následne na tlačidlo \uv{Finish}.
\end{enumerate}

Aby sme vedeli spustiť zachytávanie paketov pomocou USBPcapu cez interface Wiresharku, prejdeme do priečinku v ktorom sme nainštalovali USBPcap a skopírujeme odtiaľ súbor \uv{USBPcapCMD.exe} do priečinku \uv{extcap} v adresári kde máme nainštalovaný Wireshark.



\section{Orientácia v GUI aplikácie}
Teraz si predstavíme užívateľské rozhranie aplikácie (obrázok~\ref{obr:kap6:gui}) a funk\-cionalitu jednotlivých tlačidiel. 

\begin{figure}[!htb]
	\centering
	\includegraphics[width=\textwidth]{img/kap06_gui}
	\caption{Ukážka užívateľského rozhrania aplikácie.}
	\label{obr:kap6:gui}
\end{figure}

Popis tlačidiel je nasledovný:

\paragraph{Open} 
\hfill\break
Ako už samotný názov tlačidla napovedá, slúži na otváranie jednotlivých súborov, ktorých obsah budeme chcieť analyzovať.
Po jeho stlačení sa nám otvorí dialógové okno pomocou ktorého si zvolíme konkrétny pcap súbor. Akonáhle budeme mať daný súbor zvolený, pod tlačidlom bude vypísaný jeho názov.

\paragraph{Live Capture/File Capture}
\hfill\break
Nasleduje výber medzi Live Capture/File Capture pomocou RadioButtonov. File Capture reprezentuje analýzu ukončeného súboru, zatiaľ čo Live Capture podporuje analýzu súboru \uv{v reálnom čase}, takže je do súboru možné počas analýzy pripísať nové údaje, ktoré budú následne spracované aplikáciou.

\paragraph{Data Highlight}
\hfill\break
Je CheckBox pomocou ktorého si užívateľ zvolí, či chce mať farebne zvýraznené položky v hexdumpe na základe ich významu alebo chce mať čistý hexdump bez farebného označenia. 

\paragraph{Start}
\hfill\break
Týmto tlačidlom spustíme analýzu zvoleného súboru. Ak sa navyše jedná o File Capture, progress bar pod tlačidlami nám percentuálne ukazuje akú časť súboru už máme spracovanú. Akonáhle progress bar dosiahne 100\%, celý súbor máme úspešne spracovaný a všetky základné informácie jednotlivých paketov (index, dĺžka, typ prenosu, atď.) sa zobrazia nižšie, pričom nás to automaticky posunie na úroveň posledného paketu. V prípade Live Capture to bude vyzerať podobne, ale progress bar nám momentálne nebude ukazovať percentuálnu časť spracovania súboru (pretože tá sa môže neustále meniť). Takisto nás to po odkončení analýzy posunie na úroveň aktuálneho paketu a to sa deje vždy keď sú do analýzy pridané nové pakety.

\paragraph{Clear}
\hfill\break
Toto tlačidlo nám vyčistí plochu do ktorej sú vyobrazované základné informácie jednotlivých paketov.

\paragraph{Stop}
\hfill\break
Toto tlačidlo je navrhnuté na používanie pri Live Capture kedy ukončí čítanie aktuálneho súboru a tým znemožní vyobrazenie nových paketov. Zároveň je tým znemožnená akákoľvek ďalšia analýza súborov. Toto tlačidlo užívateľ použije v prípade, že nemá záujem o analýzu nových paketov a stačia mu tie, ktoré má momentálne zobrazené v aplikácii.

\paragraph{Pause}
\hfill\break
Tlačidlo je tatiež navrhnuté na používanie pri Live Capture, kedy ním užívateľ dáva najavo, že chce pozastaviť pridávanie nových paketov na analýzu. Po jeho kliknutí sa nápis tlačidla zmení na \uv{Continue}, čím získava opačnú funkcionalitu -- obnovenie pridávania paketov. Všetky pakety, ktoré boli v súbore v intervale medzi stlačením Pause a Continue sú vynechané a pridávanie pokračuje od nasledujúcich paketov.



\section{Používanie aplikácie}
V tejto sekcii si ukážeme prácu s aplikáciou a ako vykonať File aj Live Capture analýzu. 

\subsection{Príklad analýzy}
Momentálne si ukážeme konkrétny príklad File Capture analýzy na súbore \textit{sample.pcap}, ktorý je súčasťou prílohy k tejto práci. Budeme postupne zachytávať komunikáciu s 3 zariadeniami ukázanými na obrázku~\ref{obr:kap6:devices}:

\begin{figure}[!htb]
\centering
\begin{subfigure}{.4\textwidth}
  \centering
  \includegraphics[width=.4\linewidth]{img/genius_mys.jpg}
  \caption{Fotka genius myši prevzatá z oficiálnej genius stránky~\cite{genius_mouse_pic}.}
  \label{obr:kap6:genius:mouse:pic}
\end{subfigure}%
\begin{subfigure}{.6\textwidth}
  \centering
  \includegraphics[width=.6\linewidth]{img/kap06_joystick}
  \caption{Fotka logitech joysticku prevzatá zo stránky obchodu~\cite{logitech_joystick_pic}.}
  \label{obr:kap6:joystick_obr}
\end{subfigure}
\begin{subfigure}{\textwidth}
  \centering
  \includegraphics[width=\textwidth]{img/kap06_keyboard}
  \caption{Fotka CANYON klávesnice prevzatá zo stránky obchodu~\cite{canyon_keyboard_pic}.}
  \label{obr:kap6:keyboard_obr}
\end{subfigure}
\caption{Ukážka zariadení s ktorými budeme zachytávať komunikáciu.}
\label{obr:kap6:devices}
\end{figure}


Počas zachytávania sme vykonali nasledujúce:
\begin{enumerate}
\item Do USB portu sme pripojili zariadenie myši.
\item \label{kap06:uz_dok:krok1} Stlačili sme ľavé tlačidlo myši.
\item Stlačili sme pravé tlačidlo myši.
\item Posunuli sme koliekom myši smerom dole.
\item Vykonali sme pohyb s myšou a následne ju odpojili.
\item Pripojili sme zariadenie klávesnice.
\item Stlačili a pustili sme tlačidlo \uv{A}.
\item \label{kap06:uz_dok:krok7} Stlačili a podržali sme tlačidlo \uv{C} a potom tlačidlo \uv{D}, následne sme pustili tlačidlo \uv{C} a potom \uv{D}.
\item Stlačili a pustili sme tlačidlo \uv{Tab}.
\item Do druhého USB portu sme pripojili zariadenie joysticku.
\item Na joysticku sme stlačili a pustili tlačidlo \uv{1} a následne tlačidlo \uv{8}.
\item Stlačili a pustili sme niekoľko tlačidiel naraz.
\item \label{kap06:uz_dok:krok12} Vykonali sme pohyb analógom.
\item Ukončili sme zachytávanie.
\end{enumerate}

Teraz si otvoríme našu aplikáciu, pomocou tlačilda \uv{Open} si vyberieme súbor \textit{sample.pcap} s vyššie opísanou zachytenou komunikáciou a stlačíme tlačidlo \uv{Start}. Po načítaní súboru by sme mali byť schopní vidieť základné informácie ku všetkým paketom. V obecnom zobrazení nascrollujeme úplne hore, čím by sme mali vidieť pakety vyobrazené na obrázku~\ref{obr:kap6:uk_general}. Pre bližšiu analýzu jednotlivých paketov dvojklikneme na riadok, ktorý daný paket reprezentuje. To má za následok vytvorenie pop-up okna, v ktorom si vieme niektoré položky rozkliknúť čím vyobrazujeme detailnejšiu analýzu. Príklad pop-up okna môžeme vidieť na obrázku~\ref{obr:kap6:uk_popup}, kde sme rozklikli konkrétny Endpoint Descriptor a následne jeho položku bEndpointAddress. Každé pop-up okno má nasledujúce rozloženie:
\begin{itemize}
\item Vľavo hore máme vyobrazenú stromovú štruktúru, ktorá detailnejšie opisuje hlavičku paketu.
\item Vpravo hore je vyobrazená stromová štruktúra reprezentujúca sémantický význam zyvškových dát paketu.
\item Vľavo dole máme stromovú štruktúru opisujúcu farby v Color Map pre jenoduchšiu orientáciu v hexdumpe.
\item Vedľa Color Map sa nachádza daný hexdump, pričom tabuľka vľavo vyobrazuje znaky v ich hexadecimálnej podobe a tabuľka vpravo vyobrazuje dané znaky v ich tlačiteľnej podobe.
\end{itemize}

\begin{figure}[!htb]
	\centering
	\includegraphics[width=\textwidth]{img/kap06_uk_general}
	\caption{Ukážka obecného vyobrazenia paketov.}
	\label{obr:kap6:uk_general}
\end{figure}

\begin{figure}[!htb]
	\centering
	\includegraphics[width=\textwidth]{img/kap06_uk_popup}
	\caption{Ukážka pop-up okna aplikácie.}
	\label{obr:kap6:uk_popup}
\end{figure}

\clearpage

Momentálne si ešte ukážeme zopár obrázkov analýzy vyššie spomínanej komunikácie. V prípade, že rozklikneme paket s indexom 1, zobrazí sa nám Device Descriptor myši (obrázok~\ref{obr:kap6:uk_device_desc}). Ak si rozklikneme paket na indexe 15, mali by sme vidieť String Decriptor myši ukázaný na obrázku~\ref{obr:kap6:uk_string_desc}. Oba tieto descriptory boli poslané USB myšou počas jej konfigurácie po pripojení na zbernicu.

\begin{figure}[!htb]
	\centering
	\includegraphics[width=\textwidth]{img/kap06_device_desc}
	\caption{Ukážka Device Descriptoru myši.}
	\label{obr:kap6:uk_device_desc}
\end{figure}

\begin{figure}[!htb]
	\centering
	\includegraphics[width=\textwidth]{img/kap06_uk_string_descriptor}
	\caption{Ukážka String Descriptoru myši.}
	\label{obr:kap6:uk_string_desc}
\end{figure}

Teraz sa v obecnom zobrazení posunieme trochu nižšie a rozklikneme si paket s indexom 173, čím sa nám zobrazí analýza Report Descriptoru joystiku (obrázok~\ref{obr:kap6:uk_report_desc}).

\begin{figure}[!htb]
	\centering
	\includegraphics[width=\textwidth]{img/kap06_uk_report_desc}
	\caption{Ukážka Report Descriptoru joysticku.}
	\label{obr:kap6:uk_report_desc}
\end{figure}

Ukážeme si ešte analýzu inputu jednotlivých zariadení. V prípade myši si môžeme rozkliknúť napríklad paket s indexom 32, čo je prvý paket poslaný zariadením po jeho konfigurácii. Ak sa pozrieme vyššie na kroky, ktoré sme vykonávali počas zachytávania paketov do súboru, mal by tento paket reprezentovať krok číslo~\ref{kap06:uz_dok:krok1} čo značilo stlačenie ľavého tlačidla myši. Ukážku analýzy môžeme vidieť na obrázku~\ref{obr:kap6:uk_input_mouse}.

\begin{figure}[!htb]
	\centering
	\includegraphics[width=\textwidth]{img/kap06_uk_mouse}
	\caption{Ukážka inputu myši.}
	\label{obr:kap6:uk_input_mouse}
\end{figure}

Ak by sme si chceli pozrieť analýzu inputu klávesnice, môžeme roztvoriť napríklad paket s indexom , čo reprezentuje v našom zachytávaní krok číslo~\ref{kap06:uz_dok:krok7} -- konkrétne len jeho prvú časť, čo bola stlačenie a podržanie tlačidla \uv{C} a potom tlačidla \uv{D}. Ukážka analýzy je vyobrazená na obrázku~\ref{obr:kap6:uk_input_keyboard}.

\begin{figure}[!htb]
	\centering
	\includegraphics[width=\textwidth]{img/kap06_uk_keyboard}
	\caption{Ukážka inputu klávesnice.}
	\label{obr:kap6:uk_input_keyboard}
\end{figure}

Ako posledné sa pozrieme na input joysticku, konkrétne na krok~\ref{kap06:uz_dok:krok12} kde sme vykonali pohyb analógom. Výsledok analýzy vidíme na obrázku~\ref{obr:kap6:uk_input_joystick}

\begin{figure}[!htb]
	\centering
	\includegraphics[width=\textwidth]{img/kap06_uk_joystick}
	\caption{Ukážka inputu joysticku.}
	\label{obr:kap6:uk_input_joystick}
\end{figure}

\newpage
\hfill \break

\subsection{Vytváranie pcap súborov}
Postup vytvárania súboru sa bude líšiť vzhľadom na to o aký druh analýzy máme záujem. Obecne je intuitívnejšie pracovať v oboch prípadoch s Wiresharkom ale v prípade, že si ho nechceme nainštalovať a máme záujem využívať iba možnosť File Capture, ukážeme si aj prácu s USBPcapom.

\subsubsection{File Capture}
Na vytvorenie pcap súboru pre File Capture máme na výber použiť USBPCap alebo Wireshark.

\paragraph{USBPcap}
\hfill\break
Ak sa rozhodneme použiť USBPCap, postupujeme nasledovne:
\begin{enumerate}
\item Do tých USB portov, ktoré budeme chcieť počas zachytávania paketov sledovať pripojíme ľubovoľné HID zariadenie.
\item Zapneme USBPcap command line aplikáciu cez \textit{USBPcapCMD.exe}.
\item\label{kap6:sec:usbpcap_vyber_portov} Pomocou čísel si zvolíme USB porty, v ktorých máme zapojené zariadenia. Príklad vidíme na obrázku~\ref{obr:kap6:usbpcap_ports}, kde sú to porty 1 a 3.

\begin{figure}[!htb]
	\centering
	\includegraphics[width=12cm]{img/kap06_usbpcap_ports}
	\caption{Príklad vybratia portov 1 a 3 v aplikácii USBPcap.}
	\label{obr:kap6:usbpcap_ports}
\end{figure}

\item Následne si zvolíme meno súboru do ktorého chceme pakety zachytiť a stlačíme Enter. Vybehne nám \textit{Kontrola používateľských kont} v ktorej musíme povoliť USBPcap aplikácii vykonávanie zmien v zariadení. Predtým než to ale potvrdíme, odpojíme zariadenia zo všetkých USB portov, ktoré sme si zvolili vyššie. Toto robíme z dôvodu toho, aby sme už pri zachytávaní komunikácie dostali od zariadenia paket obsahujúci jeho HID Report Descriptor. V prípade, že zariadenia neodpojíme budeme mať síce zachytené všetky ostatné descriptory poslané počas konfigurácie, ale nebudeme schopní vykonať sémantickú analýzu inputu zariadení.
\item Po povolení vykonávaní zmien pripojíme do daných portov zariadenia, s ktorými chceme sledovať komunikáciu a môžeme s nimi vykonávať akcie, ktoré chceme mať zachytené.
\item Akonáhle chceme s analýzou skončiť, stlačíme klávesu \uv{q} a tým zároveň ukončíme USBPCap. Momentálne by sme mali nájť náš súbor v priečinku v ktorom sa nachádza \textit{USBPcapCMD.exe}.
\end{enumerate}

USBPCap má ešte jednu nevýhodu -- zachytí konfiguráciu zariadení na všetkých USB portoch (aj tých, ktoré sme si v kroku~\ref{kap6:sec:usbpcap_vyber_portov} nezvolili). Samotnú komunikáciu s nimi už ale nesleduje.

\paragraph{Wireshark}
\label{kap6:sec:wireshark:file_capture}
\hfill\break
V prípade použitia Wiresharku na vytváranie pcap súborov budeme postupovať podľa týchto krokov:
\begin{enumerate}
\item Zapneme Wireshark aplikáciu cez \textit{Wireshark.exe}.
\item Upravíme nastavenia USBPCapu stlačením na sivé koliesko vedľa jeho náz\-vu v oblasti \textit{Capture} dole (obrázok~\ref{obr:kap6:wireshark_usbpcap_settings}).

\begin{figure}[!htb]
	\centering
	\includegraphics[width=10cm]{img/kap06_wireshark_usbpcap_settings}
	\caption{Ikona nastavenia USBPcapu vo Wiresharku.}
	\label{obr:kap6:wireshark_usbpcap_settings}
\end{figure}

\item Stlačíme tlačidlo \textit{Restore Defaults} a zašktrneme možnosť \textit{Capture from newly connected devices}. Bohužiaľ, pre uloženie týchto nastavení musíme spustiť jedno zachytávanie, takže klikneme na tlačidlo \textit{Start} a potvrdíme \textit{Kontrolu používateľských kont}. Teraz môžeme toto zachytávanie zrušiť cez červený štvorec v hornej lište Wiresharku. Nastavenia sa nám týmto uložili a pokiaľ ich nebudeme chcieť zmeniť, nebudeme musieť tento krok už opakovať.
\item Teraz si vo Wiresharku cez možnosť \textit{Capture}$\rightarrow$\textit{Options} zaškrtneme v položke \textit{Input} možnosť USBPcap (obrázok~\ref{obr:kap6:wireshark_options_input}) a v položke \textit{Output} zvolíme možnosť \textit{pcap} a cez tlačidlo \textit{Browse} si vyberieme kde chceme uložiť súbor do ktorého budeme zachytávať komunikáciu (obrázok~\ref{obr:kap6:wireshark_options_output}).

\begin{figure}[!htb]
\centering
\begin{subfigure}{\textwidth}
  \centering
  \includegraphics[width=\textwidth]{img/kap06_wireshark_options_input}
  \caption{Ukážka nastavenia inputu USBPcapu vo Wiresharku}
  \label{obr:kap6:wireshark_options_input}
\end{subfigure}
\begin{subfigure}{\textwidth}
  \centering
  \includegraphics[width=\textwidth]{img/kap06_wireshark_options_output}
  \caption{Ukážka nastavenia outputu USBPcapu vo Wiresharku}
  \label{obr:kap6:wireshark_options_output}
\end{subfigure}
\caption{Ukážka nastavenia inputu a outputu USBPcapu vo Wiresharku}
\label{obr:kap6:wireshark_options_input_output}
\end{figure}

\item Skontrolujeme, že máme odpojené všetky zariadenia s ktorými chceme zachytávať komunikáciu a stlačíme tlačidlo \textit{Start}.
\item V momente keď budeme chcieť zachytávanie ukončiť, klikneme na červený štvorec v hornej lište Wiresharku.
\end{enumerate}

\subsubsection{Live Capture}
Na vykonanie Live Capture analýzy budeme potrebovať použiť aplikáciu Wireshark. Budeme postupovať pomocou rovnakých krokov ako pri File Capture~\ref{kap6:sec:wireshark:file_capture} s tou výnimkou, že teraz budeme mať počas zachytávania zapnutú aj našu aplikáciu ktorá bude vyobrazovať aktuálny stav súboru spolu s Wiresharkom.
\chapter{Záver}
\section{Zhrnutie}
celkove zhrnutie prace, ?praca s Qt?
\section{Budúce plány}




%%% Seznam použité literatury
\include{outro/literatura}

%%% Obrázky v bakalářské práci
%%% (pokud jich je malé množství, obvykle není třeba seznam uvádět)
\listoffigures

%%% Tabulky v bakalářské práci (opět nemusí být nutné uvádět)
%%% U matematických prací může být lepší přemístit seznam tabulek na začátek práce.
%\listoftables

%%% Použité zkratky v bakalářské práci (opět nemusí být nutné uvádět)
%%% U matematických prací může být lepší přemístit seznam zkratek na začátek práce.
%\chapwithtoc{Seznam použitých zkratek}

%%% Přílohy k bakalářské práci, existují-li. Každá příloha musí být alespoň jednou
%%% odkazována z vlastního textu práce. Přílohy se číslují.
%%%
%%% Do tištěné verze se spíše hodí přílohy, které lze číst a prohlížet (dodatečné
%%% tabulky a grafy, různé textové doplňky, ukázky výstupů z počítačových programů,
%%% apod.). Do elektronické verze se hodí přílohy, které budou spíše používány
%%% v elektronické podobě než čteny (zdrojové kódy programů, datové soubory,
%%% interaktivní grafy apod.). Elektronické přílohy se nahrávají do SISu a lze
%%% je také do práce vložit na CD/DVD. Povolené formáty souborů specifikuje
%%% opatření rektora č. 72/2017.
\appendix
%%% Přílohy k bakalářské práci, existují-li. Každá příloha musí být alespoň jednou
%%% odkazována z vlastního textu práce. Přílohy se číslují.
%%%
%%% Do tištěné verze se spíše hodí přílohy, které lze číst a prohlížet (dodatečné
%%% tabulky a grafy, různé textové doplňky, ukázky výstupů z počítačových programů,
%%% apod.). Do elektronické verze se hodí přílohy, které budou spíše používány
%%% v elektronické podobě než čteny (zdrojové kódy programů, datové soubory,
%%% interaktivní grafy apod.). Elektronické přílohy se nahrávají do SISu a lze
%%% je také do práce vložit na CD/DVD. Povolené formáty souborů specifikuje
%%% opatření rektora č. 72/2017.

\chapwithtoc{Prílohy}

%section{První příloha}

\openright
\end{document}
