%%% Přílohy k bakalářské práci, existují-li. Každá příloha musí být alespoň jednou
%%% odkazována z vlastního textu práce. Přílohy se číslují.
%%%
%%% Do tištěné verze se spíše hodí přílohy, které lze číst a prohlížet (dodatečné
%%% tabulky a grafy, různé textové doplňky, ukázky výstupů z počítačových programů,
%%% apod.). Do elektronické verze se hodí přílohy, které budou spíše používány
%%% v elektronické podobě než čteny (zdrojové kódy programů, datové soubory,
%%% interaktivní grafy apod.). Elektronické přílohy se nahrávají do SISu a lze
%%% je také do práce vložit na CD/DVD. Povolené formáty souborů specifikuje
%%% opatření rektora č. 72/2017.

\chapwithtoc{Prílohy}

\subsubsection{Prehľad elektronických príloh}

\dirtree{%
.1 {.}.
.2 \textbf{sample\_file} -- adresár s úkážkovým príkladom súboru so zachytenými paketami na analýzu.
.2 \textbf{html} -- html dokumentácia zdrojvých kódov vygenerovaná doxygenom.
.3 \textbf{index{.}html} -- hlavná stránka dokumentácie.
.2 \textbf{Release} -- súbory so spustiteľnou aplikáciou.
.3 \textbf{USB\_Packet\_Analyzer{.}exe} -- spustiteľná aplikácia.
.2 \textbf{src} -- adresár so zdrojovými kódmi aplikácie.
.3 \textbf{USB\_Packet\_Analyzer} -- zdrojové kódy aplikácie.
.3 \textbf{USB\_Packet\_Analyzer{.}sln} -- solution súbor pre Visual Studio 2019.
.2 \textbf{tex} -- adresár so zdrojovými kódmi textu práce pre \LaTeX.
.2 \textbf{prace{.}pdf} -- súbor obsahujúci text práce vo formáte PDF/A.
}
