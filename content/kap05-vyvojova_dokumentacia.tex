\chapter{Vývojová dokumentácia}
\section{Architekrúra aplikácie}
\section{Jadro aplikácie}
\subsection{USB\_Packet\_Analyzer}
riadi celkovy beh programu, reaguje na input od uzivatela
\subsection{Item Manager}
spracovanie samostatneho packetu a ulozenie dat o nom
\subsection{DataViewer}
trieda ktora ma na starosti vyskakovacie okno po dvojkliku a item a nasledne reaguje na input od uzivatela v okne
\subsection{TreeItem}
reprezentuje jednotlive nody v stromovej strukture ktora sa potom vyuziva na zobrazenie dat v QTreeView
\section{Modely}
\subsection{AdditionaldataModel}
model na spravovanie zvysnych dat(data ktore nie su sucastou hlavicky packetu)
\subsection{ColorMapModel}
vyobrazenie pomocnej mapy na lepsie sa zorientovanie v zvyraznemom hexdumpe
\subsection{DataViewerModel}
model na hexdump - prenasa hex/printable a zaroven o co vlastne ide(konkretny descriptor, interrupt data, ...)
\subsection{TreeItemBaseModel}
model na QTreeView ktorz vyuziva TreeItem
\subsection{USBPcapHeaderModel}
model na QTreeView ale specialne pre USBPcap hlavicku packetu
\section{Interpretery}
\subsection{BaseInterpreter}
abstractna trieda od ktorej dedia vsetkz interpretery
\subsection{Interpreter factory}
facory trieda na pridelenie konkretneho interpreteru za runtimu kvoli jednoduchosti na lepsie rozsirenie programu do buducnosti
\subsection{Interpretery descriptorov}
Config,Device,Setup,String,...
\subsection{Interrupt transfer interpretery}
obecne interrupt transfer interpreter - sluzi skor ako factory na rozne doteraz implementovane HID zariadenia
\subsubsection{Joystick interpreter}
\subsubsection{Mouse interpreter}
\subsubsection{Keyboard interpreter}
\section{Delegáti}
\subsubsection{DataViewerDelegate}
Qt delegat - stara sa o highlight hexdumpu
\section{HID}
\subsection{HIDDevices}
staticka trieda, drzi vsetky rozpoznane HID zariadenia a obsahuje funkcie specificke nich - parsovanie HID Report descriptoru
\section{Práca so súbormi}
\subsection{FileReader}
praca zo suborom a predavanie precitanych dat, offline/online capture, QFile vs std::istream
\section{Globálne dáta}
\subsection{ConstDataHolder}
staticka trieda na drzanie si konstant ktore su potrebne napriec celym programom. Mapovanie z enumu do jeho stringovej reprezentacie
\subsection{PacketExternStructs}
obsahuje definiciu vsetkych dolezitych USBPcap structov, pcap structov, enumov a vsetkych structov ktore pouzivam v aplikacii











