\chapter{Vývojová dokumentácia}
\label{chap:vyvoj_dok}

Táto kapitola je zameraná pre programátorov, ktorých by bližšie zaujímali implementačné detaily a štruktúra nášho programu. Najprv si opíšeme ako skompilovať priložené zdrojové súbory práce a neskôr sa pozrieme na stavbu programu.

\section{Kompilácia}
V prílohe tejto práce sa nachádzajú zdrojové kódy nášho programu. Na ich úspešné skompilovanie budeme ale potrebovať splniť niekoľko požiadaviek:
\begin{itemize}
\item Podporovaná platforma na kompiláciu je momentálne iba Windows 10.
\item Ku kompilácii je potrebné Visual Studio 2019 s SDK verziou 10.0.19041.0
\item Qt verzie 5.15.0
\end{itemize}

\subsection{Inštalácia Qt}

Keďže ku kompilácii budeme potrebovať samotné Qt, ukážeme si jeho inštaláciu v nasledujúcich krokoch (v prípade, že náš počítač už obsahuje požadovanú verziu Qt, môžeme tieto kroky preskočiť a prejsť na integráciu VS s Qt~\ref{kap4:sec:VS2019_qt}). Tu je nutné spomenúť, že sa jedná o online inštaláciu a budeme potrebovať pripojenie k internetu:
\begin{itemize}
\item Po kliknutí  \href{https://www.qt.io/download}{\textbf{tu}} sa nám otvorí internetový prehliadač so stránkou na stiahnutie Qt, kde vyhľadáme možnosť \uv{Downloads for open source users} a klikneme na \uv{Go open source} (ukázané na obrázku~\ref{obr:kap4:qt_download}).

\begin{figure}[!htb]
	\centering
	\includegraphics[width=12cm]{img/kap04_qt_download}
	\caption{Stránka Qt download.}
	\label{obr:kap4:qt_download}
\end{figure}

\item Na aktuálnej stránke nascrollujeme úplne dole, kde klikneme na zelené tlačidlo s nápisom \uv{Download the Qt Online Installer} (obrázok~\ref{obr:kap4:qt_online}).

\begin{figure}[!htb]
	\centering
	\includegraphics[width=12cm]{img/kap04_qt_online}
	\caption{Stránka Qt -- odkaz na stiahnutie \uv{Qt Online Installer}.}
	\label{obr:kap4:qt_online}
\end{figure}

\item Na aktuálnej stránke klikneme na \uv{Download}, čím sa nám stiahne \uv{Qt Online Installer}. Ten následne spustíme. Ukážku vidíme na obrázku~\ref{obr:kap4:qt_online_down}.

\begin{figure}[!htb]
	\centering
	\includegraphics[width=12cm]{img/kap04_qt_online_down}
	\caption{Stránka Qt stiahnutie uv{Qt Online Installeru}.}
	\label{obr:kap4:qt_online_down}
\end{figure}

\newpage
\newpage

\item Ako prvé nás privíta obrazovka, ktorá od nás vyžaduje prihlásenie sa do nášho Qt účtu. Toto je bohužiaľ nutnosť a nie je možné bez toho pokračovať. V prípade, že nemáme žiadny vytvorený Qt účet, môžeme si ho vytvroiť priamo počas inštalácie a celé to nezaberie viac ako dve minúty. Potom klikneme na tlačidlo \uv{Next} (obrázok~\ref{obr:kap4:inst_welcome}).

\begin{figure}[!htb]
	\centering
	\includegraphics[width=12cm]{img/kap04_inst_welcome}
	\caption{\uv{Welcome} časť Qt Setupu}
	\label{obr:kap4:inst_welcome}
\end{figure}


\item Teraz zaškrtneme, že sme si prečítali a súhlasíme s povinnosťami používania \uv{Open Source Qt} a klikneme na tlačidlo \uv{Next} (obrázok~\ref{obr:kap4:inst_oblg}).

\begin{figure}[!htb]
	\centering
	\includegraphics[width=12cm]{img/kap04_inst_oblg}
	\caption{\uv{Open Source Obligations} časť Qt Setupu}
	\label{obr:kap4:inst_oblg}
\end{figure}

\item Momentálne len klikneme na tlačidlo \uv{Next} a počkáme kým sa nám stiahnu dáta z repozitára (obrázok~\ref{obr:kap4:inst_set}).

\begin{figure}[!htb]
	\centering
	\includegraphics[width=12cm]{img/kap04_inst_set}
	\caption{\uv{Setup - Qt} časť Qt Setupu}
	\label{obr:kap4:inst_set}
\end{figure}

\item Teraz zaškrtneme jednu z možností o zasielaní štatistických údajov počas používania Qt Creatoru. Tu si môžeme zvoliť ktorúkoľvek z možností podľa osobných preferencií. Následne klikneme na tlačidlo \uv{Next} (obrázok~\ref{obr:kap4:inst_cont}).

\begin{figure}[!htb]
	\centering
	\includegraphics[width=12cm]{img/kap04_inst_cont}
	\caption{\uv{Contribute to Qt Development} časť Qt Setupu}
	\label{obr:kap4:inst_cont}
\end{figure}

\item Na tejto obrazovke (obrázok~\ref{obr:kap4:inst_int}) si zvolíme adresár do ktorého chceme Qt nainštalovať a zaklikneme možnosť \uv{Custom installation}. Následne klikneme na \uv{Next}.

\begin{figure}[!htb]
	\centering
	\includegraphics[width=12cm]{img/kap04_inst_int}
	\caption{\uv{Installation Folder} časť Qt Setupu}
	\label{obr:kap4:inst_int}
\end{figure}

\item Teraz si zvolíme konkrétne komponenty, ktoré sa nám nainštalujú. Komponenty, ktoré sú zaškrtnuté necháme tak a rozklikneme možnosť Qt$\leftarrow$Qt 5.15.0 a zaškrtneme možnosť \uv{MSVC 2019 64-bit} a klikneme na \uv{Next} (obrázok~\ref{obr:kap4:inst_sel}).

\begin{figure}[!htb]
	\centering
	\includegraphics[width=12cm]{img/kap04_inst_sel}
	\caption{\uv{Select Components} časť Qt Setupu}
	\label{obr:kap4:inst_sel}
\end{figure}

\item Na aktuálnej obrazovke (obrázok~\ref{obr:kap4:inst_lic}) zaškrtneme možnosť, že sme si prečítali a súhlasíme s licenčnej zmluvy a klikneme na \uv{Next}.

\begin{figure}[!htb]
	\centering
	\includegraphics[width=12cm]{img/kap04_inst_lic}
	\caption{\uv{License Agreement} časť Qt Setupu}
	\label{obr:kap4:inst_lic}
\end{figure}

\item Zvolíme si start menu skratku a klikneme na \uv{Next} (obrázok~\ref{obr:kap4:inst_star}).

\begin{figure}[!htb]
	\centering
	\includegraphics[width=12cm]{img/kap04_inst_star}
	\caption{\uv{Start Menu Shortcut} časť Qt Setupu}
	\label{obr:kap4:inst_star}
\end{figure}

\item Teraz už len klikneme na \uv{Install} (obrázok~\ref{obr:kap4:inst_ready}) a počkáme kým prebehne inštalácia.

\begin{figure}[!htb]
	\centering
	\includegraphics[width=12cm]{img/kap04_inst_ready}
	\caption{\uv{Ready to Install} časť Qt Setupu}
	\label{obr:kap4:inst_ready}
\end{figure}

\item V záverečnej obrazovke môžeme odškrtnúť možnosť \uv{Launch Qt Creator} a klikneme na tlačidlo \uv{Finish}.

\clearpage

\subsection{Visual Studio 2019 a Qt}
\label{kap4:sec:VS2019_qt}
Teraz si ešte ukážeme integrovanie Visual studia 2019 spolu s Qt. Budeme si potrebovať nainštalovať \textit{Qt VS Tools}. Podrobný postup si ukážeme v nasledujúcich krokoch:
\begin{itemize}
\label{kap4:qt_vs_integ}
\item Priamo vo Visual Studiu si cez možnosť \textit{Extensions}$\rightarrow$\textit{Manage Extensions} do online hľadania zadáme \uv{Qt} a stiahneme si rozšírenie \uv{Qt Visual Studio Tools}~\cite{qt_vs_tools}
\item\label{kap4:qt_vs_integ:krok2} Po úspešnej inštalácii si musíme ešte vybrať Qt verziu cez možnosť \textit{Extensions}$\rightarrow$\textit{Qt VS Tools}$\rightarrow$\textit{Qt Versions} (ukázané na obrázku~\ref{obr:kap4:vs_versions}).

\begin{figure}[!htb]
	\centering
	\includegraphics[width=12cm]{img/kap04_vs_versions}
	\caption{Visual Studio možnosť vybratia Qt verzie.}
	\label{obr:kap4:vs_versions}
\end{figure}

\item V dialógu sa teraz cez ikonu v stĺpčeku \uv{Path} (obrázok~\ref{obr:kap4:vs_path}) odkážeme do adresáru, kde sme nainštalovali \textit{Qt} a následne do adresáru \textit{5.15.0\/msvc2019\_64\/bin} kde máme nainštalovaný \textit{qmake.exe}.

\begin{figure}[!htb]
	\centering
	\includegraphics[width=12cm]{img/kap04_vs_path}
	\caption{Visual Studio zvolenie adresára ku \textit{qmake.exe}.}
	\label{obr:kap4:vs_path}
\end{figure}

\item Ako posledné si ešte skontrolujeme cez možnosť \textit{Project}$\rightarrow$\textit{Properties} v \textit{Configuration Properties}$\rightarrow$\textit{General} skontrolujeme, že máme nastavený \textit{C++ Language Standard} na možnosť \textit{ISO C++ 17} a \textit{Windows SDK Version} na verziu \textit{10.0.19041.0} ako na obrázku~\ref{obr:kap4:vs_prop}.

\begin{figure}[!htb]
	\centering
	\includegraphics[width=12cm]{img/kap04_vs_prop}
	\caption{Visual Studio obecné nastavenia projektu.}
	\label{obr:kap4:vs_prop}
\end{figure}

\end{itemize}
Môže sa stať, že narazíme na bug keď nám krok~\ref{kap4:qt_vs_integ:krok2} nenastaví Qt verziu a nebudeme vedieť projekt preložiť kvôli errorom typu \uv{cannot open source file qbytearray}. V takom prípade prejdeme do možnosti \textit{Project}$\rightarrow$\textit{Properties}$\rightarrow$\textit{Qt Project Settings} a manuálne nastavíme \textit{Qt Installation} na \textit{msvc2019\_64} (obrázok~\ref{obr:kap4:vs_manual}).

\begin{figure}[!htb]
	\centering
	\includegraphics[width=12cm]{img/kap04_vs_manual}
	\caption{Visual Studio obecné nastavenia projektu.}
	\label{obr:kap4:vs_manual}
\end{figure}

V tomto momente by sme mali byť schopní úspešne skompilovať a spustiť náš program.
%Budeme si potrebovať nainštalovať Qt pre Windows. podla videa, verzia 5.15.0 len msvc 2019 64bit. inak ostatne default. zatial som este nainstaloval QT VS extension. teraz idem updatenut SDK na verziu 10.0.19041.0 -- to vymazalo errory typu ''cannot open source file cctype'' , teraz mi este ostali ''cannot open source file qbytearray'' a podobne spojene cisto s qt : pravdepodobne bug. musel som ist manualne do Project->properties->Qt Project SEttings a nastavit Qt installation na verziu 5.15.0_msvc2019_64. A UZ TO BEZIIII :)
\end{itemize}



























\newpage
\section{Architekrúra aplikácie}
\section{Jadro aplikácie}
\subsection{USB\_Packet\_Analyzer}
riadi celkovy beh programu, reaguje na input od uzivatela
\subsection{Item Manager}
spracovanie samostatneho packetu a ulozenie dat o nom
\subsection{DataViewer}
trieda ktora ma na starosti vyskakovacie okno po dvojkliku a item a nasledne reaguje na input od uzivatela v okne
\subsection{TreeItem}
reprezentuje jednotlive nody v stromovej strukture ktora sa potom vyuziva na zobrazenie dat v QTreeView
\section{Modely}
\subsection{AdditionaldataModel}
model na spravovanie zvysnych dat(data ktore nie su sucastou hlavicky packetu)
\subsection{ColorMapModel}
vyobrazenie pomocnej mapy na lepsie sa zorientovanie v zvyraznemom hexdumpe
\subsection{DataViewerModel}
model na hexdump - prenasa hex/printable a zaroven o co vlastne ide(konkretny descriptor, interrupt data, ...)
\subsection{TreeItemBaseModel}
model na QTreeView ktorz vyuziva TreeItem
\subsection{USBPcapHeaderModel}
model na QTreeView ale specialne pre USBPcap hlavicku packetu
\section{Interpretery}
\subsection{BaseInterpreter}
abstractna trieda od ktorej dedia vsetkz interpretery
\subsection{Interpreter factory}
facory trieda na pridelenie konkretneho interpreteru za runtimu kvoli jednoduchosti na lepsie rozsirenie programu do buducnosti
\subsection{Interpretery descriptorov}
Config,Device,Setup,String,...
\subsection{Interrupt transfer interpretery}
obecne interrupt transfer interpreter - sluzi skor ako factory na rozne doteraz implementovane HID zariadenia
\subsubsection{Joystick interpreter}
\subsubsection{Mouse interpreter}
\subsubsection{Keyboard interpreter}
\section{Delegáti}
\subsubsection{DataViewerDelegate}
Qt delegat - stara sa o highlight hexdumpu
\section{HID}
\subsection{HIDDevices}
staticka trieda, drzi vsetky rozpoznane HID zariadenia a obsahuje funkcie specificke nich - parsovanie HID Report descriptoru
\section{Práca so súbormi}
\subsection{FileReader}
praca zo suborom a predavanie precitanych dat, offline/online capture, QFile vs std::istream
\section{Globálne dáta}
\subsection{ConstDataHolder}
staticka trieda na drzanie si konstant ktore su potrebne napriec celym programom. Mapovanie z enumu do jeho stringovej reprezentacie
\subsection{PacketExternStructs}
obsahuje definiciu vsetkych dolezitych USBPcap structov, pcap structov, enumov a vsetkych structov ktore pouzivam v aplikacii











