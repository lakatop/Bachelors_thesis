\chapter{Analýza}
\section{Získanie USB packetov}
\subsection{Windows exclusive mód}
opisat co to je, a dolezite je spomenut, ze windows otvara v exclusive mode zakladne HID zariadenia ako mys a klavesnica
\subsection{Známe knižnice}
opisat zakladne kniznice na sledovanie USB zbernice a preco som ich nemohol pouzit : libUSB, hidAPI, moufiltr, SetupAPI, WinUSB
\subsection{Third-party aplikácie}
opisat odkial nakoniec ziskavam packety - USBPcap a Wireshark
\section{Spracovávanie pcap súborov}
moznosti ako citat pcap subory : bud pouzit uz existujucu kniznicu : na linuxe Libpcap, windows NPcap(deprecated WinPcap), alebo citat subory manualne : std::istream alebo QFile
\section{Sémantická reprezentácia dát}
ako si z dat vytiahnut udaje ktore su potom pouzite na semanticku analyzu implementovanych HID zariadeni : HID Report parser, InputValues a EndpointDevice struct.
Nasledne sparovanie - ako vybrat spravny report pre konkretny input
\section{Voľba frameworku}
obecne co by som od toho GUI priblizne chcel, potom opisat preco som si vybral prave Qt a v nasledujucich kapitolach opisat rozhodnutia uz v Qt
dovod preco som si zvolil qt namiesto inych c++ GUI frameworkov(napriklad sfml)
\section{Zobrazenie základných informácií}
ako zobrazovat zakladne info o packete : pouzit QListWidget alebo QTableWidget (pripadne nieco ine ako nejaky abstract viewmodel), narok na zakladne funkcionality : lahka rozsiritenlnost o dalsie ''stlpceky'' , moznost jednoduchej interakcie(doubleClick na polozku). Mat vsetky info na jednom okne / mat pop-up okna.
\section{Zobrazenie sémantického významu dát}
ako vyzobrazit semanticky vyznam roznych dat - descriptory, usb header, vyznam input dat roznych HID zariadeni
\section{Hexdump}
ako v qt urobit hexdump - do coho zobrazovat data(vytvorit si vlastny viewer dedeny od QAbstractScrollArea, pripadne niecoho ineho) vs najst nieco co uz v qt je a upravit to aby to sedelo poziadavkam. Vziat do uvahy bezne funkcie hexdumpu : selection mody(oznacit naraz hexa a im odpovedajuce printable), logicke oddelenie dat(napriklad farbami)








